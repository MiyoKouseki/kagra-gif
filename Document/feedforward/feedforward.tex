\abstract{}


\section{アームにGIFの信号をFeedForwardすれば安定になるか?}

\subsection{制御のおさらい}
制御の目的はシステムの出力値を目標値に近づけることである。そのための方法としてFeedback制御とFeedforward制御の2つを考える。
\subsubsection{安定とは}
ここで問題にしたい安定性は,FBの安定性というより,定常偏差のことかもしれない。いくらUGFあげて位相余裕があっても,定常偏差が大きいと,腕のロックができない。

\subsubsection{2自由度PID制御}

FeedBack制御だけだと外乱抑制と目標追従の性能の両立はできないが,FeedForwardをつかえばできる\cite{araki2003two}。


\subsection{LIGOの例}
\subsubsection{FBとFF}
IPに置いた地震計でFeedbackするのと,地面に置いた地震計でFeedForwardすることの違いを述べる。

LIGOではプラットフォームにおいた地震計によるFeedfbackと,地面に置いた地震計をつかったFeedforwardがある。

\subsubsection{Sensor Correction}



\subsubsection{潮汐}
TidalはDC制御している。おそらく潮汐の周波数でゲインをもつようなローパスフィルタをかけてプラットフォームに戻しているはず。これって制御を不安定にしない??してない。


\subsection{環境変動による不安定性}
地震がきたときに高周波がノイジーでいいから,地震の影響を受けないようにフィルターを変える試みがある\cite{biscans2018control}。

この原因は?
\subsubsection{地面振動の変動}
地震がくると30mHzあたりで脈動よりもRMSがおおきくなる。


\subsection{KAGRAでできること}
\subsubsection{GIFと地震計の比較}
\subsubsection{制御のプラン}


\bibliography{./feedforward} %hoge.bibから拡張子を外した名前
\bibliographystyle{junsrt} %参考文献出力スタイル


