\documentclass[a4paper,12pt]{jsarticle}
\bibliographystyle{junsrt}
\usepackage{ascmac}
\usepackage{empheq}
\usepackage{amsmath,amssymb}
\usepackage{bm}
\usepackage[dvipdfmx]{graphicx,color}
\usepackage[top=30truemm,bottom=30truemm,left=30truemm,right=30truemm]{geometry}
\usepackage[format=hang,margin=75pt,font=small]{caption}
\usepackage{here}
\usepackage{comment}
\title{レーザー歪み計を用いた能動防振の提案}
\author{三代浩世希}
\begin{document}
\setcounter{tocdepth}{3}
\maketitle
\abstract{能動防振は0.1Hz以下の低周波では慣性系に対して防振できない。正確に言えば変位センサーでDC制御を行っているためローカルの地面に対して防振している。この状態で腕共振器を組むと、基線長が短い場合は低周波地面振動は同相成分として現れるため問題になりにくいが、kmスケールでは逆相成分である基線長変動に現れてしまうことで、lock aquisition が不安定になる。LIGO では sensor correction と呼ばれる手法で変位センサーから地面振動成分を取り除き、慣性系に対して防振できるよう補正する試みがなされている。しかし慣性センサーは低周波で感度が悪くなることや、tilt-horizontal coupling による傾斜成分のカップリングなどによって、原理的に低周波防振には限りがある。本提案ではsensor correction のセンサーにGIFのレーザー歪み計をつかう。レーザー歪み計は基線長伸縮を原理的にはDCまで直接測定することが可能であり、基線長が一定になるように変位センサー信号を補正することができる。腕共振器の低周波地面振動による不安定性を改善することにより、ロックロスを低減することが期待される。}
\tableofcontents
\chapter{Geophysics Interferometer}
\section{Overview}
\section{Purpose}
\subsection{Motivation in Geophysics}
\subsection{Motivation in GW detectors}
\section{Working Principle}
\subsection{Response to the seismic strain}
\subsection{Signal detection Scheme}
\subsection{Noise}
\section{Optics}
\subsection{Mode Matching Optics}
\subsection{Frequency Stabilized Laser}
\subsection{Core Optics}
\section{Data Aquisition System}
\subsection{Stand Alone System}
\subsection{Realtime System}
\subsection{...}
\section{Summary of the Chapter}

\bibliography{./reference}

\end{document}
