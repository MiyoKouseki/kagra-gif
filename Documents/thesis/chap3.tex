\chapter{Underground Seismic Noise} %地下の地面振動




\section{Introduction} %イントロ
\subsection{KAGRA Tunnel}
KAGRAのトンネルについて述べる。センターと両エンドが地表からどれぐらいの深さにあるかとか、どういう岩石が分布しているとか。たしか三代木さんがトンネルの図を持っていた。




\section{Theory of Seismic Waves} %弾性波理論
本章の説明で必要になる理論を述べる。
\subsection{Seismic Waves}
\subsubsection{Body Waves}
\subsubsection{Surface Waves}
\subsection{Spatial Autocorrelation}
\subsection{Common and Differential Mode Ratio (CDMR)}





\section{Seismic Noise} %KAGRAのサイト
\subsection{Long Term Characteristics}
\subsection{Microseismic Peak Model}
\subsection{Peak identification}





\section{Fluctuation of the Arm Length} %基線長伸縮
KAGRAは、硬い岩盤おかげで基線長伸縮が他のサイトよりもよく低減されていることを述べる。
\subsection{}



\section{Summary of the Chapter} %まとめ
