\documentclass[a4paper,12pt]{jsarticle}
\bibliographystyle{junsrt}
\usepackage{ascmac}
\usepackage{empheq}
\usepackage{amsmath,amssymb}
\usepackage{bm}
%\usepackage{pxjahyper}
%\usepackage[dvipdfmx]{graphicx}
\usepackage[dvipdfmx]{graphicx,color}
\usepackage[top=30truemm,bottom=30truemm,left=30truemm,right=30truemm]{geometry}
\usepackage[format=hang,margin=75pt,font=small]{caption}
%\usepackage{physics}
\usepackage{braket}
\usepackage{here}
\usepackage{comment}
\title{どれだけ基線長伸縮を小さくすべきか}
\author{三代浩世希}
\begin{document}
\setcounter{tocdepth}{2}
\maketitle
\abstract{
}
\section{Rquirement on the control}
\subsection{Mirror Velocity Requirement}
Greenロックを使わない場合。
\subsubsection{Bandwidth Limit}
制御力が十分大きい場合を考える。鏡が共振幅に入ってから出るまでの時間に制御がかかれば、引き込みは可能である。その時間は、線幅を$L\,\mathrm{[m]}$通過時の鏡の速度を$v\, \mathrm{[m/sec]}$とすれば、$L/v\, \mathrm{[sec]}$で与えられる。また外乱抑制にかかる時間はおそよバンド幅の逆数で与えられるので、\footnote[1]{フィードバック制御の基礎 p. 146 に書かれているステップ応答の立ち上がり時間を引用した。今考えている共振器の引き込みの場合、外乱のインパルス応答を考えるのが正しい。けれど多分ステップ応答でもインパルス応答でも、立ち下がり時間はバンド幅の逆数で与えられるはず。計算していないけど。}鏡の速度は以下の条件式で制限される。
\begin{equation} \label{eq:eq01}
v < L \omega_{b}
\end{equation}
しかし条件式(\ref{eq:eq01})は十分大きな制御信号が印加できる場合でのみ正しい。実際のアクチュエータ効率には限りがある。アクチュエータ効率を大きくするとアクチュエータからノイズが流入し、重力波感度を汚すためである。
\subsubsection{Actuation Power Limit}
つぎに制御力に限りがある場合を考える。その場合、アクチュエータが鏡に与える仕事で、運動している鏡を線幅以内で静止するためには
\begin{equation} \label{eq:eq02}
\frac{1}{2}mv^2 < F L
\end{equation}
を満たさなければならない。ここで$m\,\mathrm{[kg]}$は鏡の質量、$F\,\mathrm{[N]}$はアクチュエータが鏡に与える力である。制御力$F$には先述したように上限があるので、$F=F_{\mathrm{max}}=AV_{\mathrm{max}}$とすれば、
\begin{equation} \label{eq:eq03}
v < \sqrt{\frac{2AV_{\mathrm{max}}L}{m}}
\end{equation}
のように、引き込み可能な鏡の速さはアクチュエータ効率$A\,\mathrm{[N/V]}$の大きさで決まる。ここで$V_{\mathrm{max}}\,\mathrm{[V]}$は印加可能な電圧値であり、ADCの最大出力で決まる。

アクチュエータ効率は鏡の変位雑音を汚さないよう十分小さい値を取らなければならない。つまり以下のような条件になる。
\begin{equation} \label{eq:eq04}
  A V_n  \frac{1}{\omega^2} < X_{\mathrm{req}}
\end{equation}
ここで、$X_{\mathrm{req}}$は鏡の変位雑音の要求値で$V_n$はDACノイズである。式(\ref{eq:eq04})に、これら値を代入すると、アクチュエータ効率$A\,\mathrm{[N/V]}$は〇〇となる。

結果として、引き込み可能な鏡の速度の上限は()と見積もることができる。
\subsection{RMS Velocity Requirement}
Greenロックを使う場合。
\subsubsection{RMS Velocity Requirement}


\end{document}

