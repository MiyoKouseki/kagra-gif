\chapter{Geophysics Interferometer (GIF)}




\section{Overview} % この章について説明する。
項の関係をここで述べる。




\section{Purpose} % GIFの目的
この干渉計の目的を述べる。どういう地球物理の現象をターゲットにしているかとか述べる。それをもとに要求値が決まっていく。
\subsection{Motivation in Geophysics}
地物でのモチベーション。(ここは新谷さん高森さん早河さんに聞いて書く。)
\subsection{Motivation in GW detectors}
重力波望遠鏡でのモチベーション。Drever氏のSPIの原論文で述べられていた「Seismic Interferometer」のように基線長伸縮を低減することが、GIFをKAGRAに設置するモチベーション。




\section{Working Principle} %動作原理 
動作原理を書く。どうやって地面の歪みをマイケルソン干渉計で検出するのかここに書く。
\subsection{Response to the seismic strain}
どうやって地面の歪みが基線長伸縮として応答するか述べる。地面の歪みから基線長伸縮への伝達関数を載せる。
\subsection{Signal Detection Scheme}
どうやって干渉計信号から基線長伸縮を取り出すか述べる。Quadrature Phase Detection について書く。ここからサンプリング周波数への要求値が定まることを述べる。
\subsection{Noise}
どういうノイズが原理的に存在するか述べる。空気ゆらぎ、周波数雑音を述べる。




\section{Optics} %光学系
どうやって実際の干渉計を構築しているか述べる。
\subsection{Mode Matching Optics}
どういうモードマッチをして干渉計として光を干渉させているか述べる。
\subsection{Frequency Stabilized Laser}
どういう制御をして周波数安定をしているか述べる。
\subsection{Core Optics}
\subsubsection{Beam Splitter}
どういうミラーを使っているか述べる。
\subsubsection{Corner Cube}
どういうミラーを使っているか述べる。大きさとか表面の精度とか。




\section{Data Aquisition System} %DAQ
DAQについて述べる。冗長性を持たせるために二系統のDAQを使っていることを述べる。一方はKAGRAとは独立で、もう一方はKAGRAと同じシステムに組み込んでいることを述べる。
\subsection{Stand Alone System}
森井システムについてのべる。コンパクトなシステムで地下環境でも安定して動くシステムだ、と述べる。
\subsection{Realtime System}
KAGRAのリアルタイムシステムについて述べる。KAGRAの制御に組み込むために歪変換をリアルタイムで行っている、と述べる。




\section{Summary of the Chapter} %章のまとめ
本章で述べたパラメータを表にまとめる。
