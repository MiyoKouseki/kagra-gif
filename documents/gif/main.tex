\documentclass[a4paper,12pt]{book}
\bibliographystyle{junsrt}
\usepackage{ascmac}
\usepackage{empheq}
\usepackage{amsmath,amssymb}
\usepackage{bm}
\usepackage[dvipdfmx]{graphicx,color,hyperref}
\usepackage[top=30truemm,bottom=30truemm,left=30truemm,right=30truemm]{geometry}
\usepackage{braket}
\usepackage{here}
\usepackage{comment}
\usepackage{jtygm} % To avoid warning
\usepackage[hang,small,bf]{caption}
\usepackage[subrefformat=parens]{subcaption}
\captionsetup{compatibility=false}
\title{Geophysics Interferometer}
\author{Koseki Miyo}

% ハイパーリンクを付ける設定
\usepackage{pxjahyper}
\hypersetup{
  colorlinks=true,
  linkcolor=blue,
  %bookmarks=true, 
  bookmarksnumbered=true,
  pdfborder={0 0 0},
  bookmarkstype=toc
}


\begin{document}
\setcounter{tocdepth}{2}
\maketitle

\tableofcontents
\chapter{Geophysics Interferometer}
\section{Overview}
\section{Purpose}
\subsection{Motivation in Geophysics}
\subsection{Motivation in GW detectors}
\section{Working Principle}
\subsection{Response to the seismic strain}
\subsection{Signal detection Scheme}
\subsection{Noise}
\section{Optics}
\subsection{Mode Matching Optics}
\subsection{Frequency Stabilized Laser}
\subsection{Core Optics}
\section{Data Aquisition System}
\subsection{Stand Alone System}
\subsection{Realtime System}
\subsection{...}
\section{Summary of the Chapter}

\appendix

\end{document}
