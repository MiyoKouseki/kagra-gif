\section{山の歪み}
回転しない場合,ひずみテンソルは垂直ひずみ成分とせん断ひずみ成分に分けることができる。またさらに垂直ひずみ成分は,同相成分と逆相成分とに分けることができる。つまり,
\begin{equation}\label{eq:eq43}
  aa
\end{equation}
となる。


\subsubsection{同相成分}
同相成分は,すべての軸が同じ大きさで伸縮するひずみであり,体積ひずみや面積ひずみともよばれている。そのため,同相成分であれば,Xアームで計測したものはYアームでも同じになると言える。この同相成分はKAGRAのメインレーザーの周波数安定化に使われる。


一様等方性を仮定すれば,同相成分はおもに大気圧による等方圧縮もしくは等方膨張からなる。\footnote[6]{しかし,一様等方な岩盤であれば同相成分のみに寄与するが,非等方性がある山の中の場合,他の逆相成分やせん断ひずみ成分にもカップリングすると思われる。そのた有限要素法で地形をモデリングする必要がある。これは最低限D論の内容にふくめたい。他の成分に}


図()に大気圧とGIFのコヒーレンスを示す。
およそ数10mHz以下ではコヒーレンスは有意であり,大気圧応答をみていると考えられる。このときの気圧ひずみ応答の係数は○○であり,▲▲と矛盾しない結果となっ。したがって,この帯域ではYアームの推定に使える。



\subsubsection{逆相成分}
逆相成分は,ある一方の軸が膨張したとき,他方の面が同じ大きさで伸縮するひずみである。逆相成分には重力波信号が含まれている。


岩盤の非一様性や非等方性などにより,他の同相成分やせん断ひずみ成分からカップルする。


\subsubsection{せん断成分}
せん断成分は,干渉計の基線長変化には影響しない。鏡の角度揺れに対応するが,多分影響はちいさいので無視する。\footnote[7]{ほんとうか?}

\subsubsection{}


本質的に,自由度が1つしかないので,逆相成分はわからない。


自由度が1つしかないので,せいぜい$\sqrt{2}$程度しか低減できない。(とはいってもGIFのノイズレベルまで小さくできるはず。変位換算で$10^{-10}\, \mathrm{m}$までは。)


まずは,Xアームの安定化を考える。

