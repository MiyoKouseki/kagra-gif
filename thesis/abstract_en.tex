\clearpage
%\chapter*{Abstract} \addcontentsline{toc}{chapter}{Abstract}
{\huge \bf Abstract} \\

In 2015, two LIGO detectors had directly detected the gravitational wave (GW) from the black-hole binary merger event, GW150914. In 2017, three detectors, including the Virgo detector, had detected the GW from the neutron star merger, GW170817. Moreover, the electromagnetic counterpart was identified by the follow-up observations. Multi-messenger astronomy was established from this time.

GW observation needs a coincidence detection with multiple GW detectors because a single detector cannot determine the direction of the GW source. If multiple detectors detect the GW, we can estimate the direction from the differences of these detection times. To determine the direction, at least three GW detectors are needed. However, the duty cycle of the GW detectors is almost 60\%, and that of multiple detectors is below 50\%.

This duty cycle is limited by the unstable operation of GW detectors' unstable operation, caused by a seismic disturbance mainly below 1 Hz. While these seismic noises disturb the baseline in this frequency, the current vibration isolation system for GW detectors does not have the isolation performance in this low-frequency seismic noise because of the insufficient sensitivity of the inertial sensor used in this isolation system.

In this study, the baseline compensation system has been developed.  Unlike the current vibration isolation system, this system uses a 1500 m strainmeter installed in parallel KAGRA baseline, which is named the geophysics interferometer (GIF). The GIF is designed and developed for monitoring the deformation of the baseline directly below 1 Hz with high sensitivity. For this reason, we designed the new system to compensate for the baseline so that the KAGRA interferometer would not be affected by low-frequency seismic disturbances.

In this thesis, two main topics are described: the influence of the seismic disturbances on GW detectors and the baseline compensation system needed to attenuate these disturbances. The design, the performance, and the advantage of this new system are described. Moreover, the implementation and demonstration of this system on the KAGRA interferometer are also described. In the test demonstration, the baseline compensation system is installed on the X-arm cavity, which is the most sensitive component in the GW detectors. As a result, the cavity length fluctuation caused by the deformation of the baseline is reduced by -6 dB above 0.01 Hz and by -40 dB below this frequency.

This result would increase the duty cycle of the KAGRA interferometer. If this new vibration isolation system can be installed in other GW detectors, the coincidence duty cycle can also be improved. 

