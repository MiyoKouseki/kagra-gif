\clearpage
%\chapter*{Abstract} \addcontentsline{toc}{chapter}{Abstract}
{\huge \bf Abstract} \\

In 2015, two LIGO detectors had directly detected the gravitational-wave (GW) from the black hole binary marger event, GW150914. In 2017, three detectors including Virgo detector had detected the GW from the neutron star binary merger evernt, GW170817. In addition, the electromagnetic counterpart was identified by the follow-up observations, and the multi-messenger astronomy was established from this time. Moreover, in 2020, four GW detectors including KAGRA detector will detect further GW events.

Although the coincidence observation is required, the duty cycle of the laser interferometric GW detector is almost $60\,\%$. This unstable operation is caused by the lock loss of the interferometer because of the large seismic noises such as the long-period earthquakes or the noise associated with the ocean waves in the unfavorable] weather condition. While these seismic noises disturb the baseline mainly below $1\,\mathrm{Hz}$, the current vibration isolation system for GW detectors does not have the isolation performance in these low-frequency seismic noise, because of the insufficient sensitivity of the inertial sensor used in this isolation system.

  In this study, the baseline compensation system have developed. Unlike the current vibration isolation system, this system uses a 1500 m strainmeter installed in parallel KAGRA baseline, which is named geophysics interferometer (GIF). GIF is developed for monitoring the deformation of the baseline directly below 1 Hz with high sensitivity. For this reason, we designed the new system to compensate the baseline so that KAGRA interferometer would not be affected by low-frequency seismic disturbances.

  In this thesis, two main topics are described: influence of the seismic disturbances to GW detectors and the baseline compensation system to attenuate these disturbances. Whereas the design, the performance, and the advantage of this new system are described, the implementation and demonstration of this system on KAGRA interferometer are also described. In the test demonstration, the baseline compensation system is installed on the X-arm caivty, which is the most sensitive components in the GW detectors. As a reslut, the cavity length fluctuation caused by the deformation of the baseline is reduced by -6 db above 0.01 Hz and by -20 db below this frequency.  
