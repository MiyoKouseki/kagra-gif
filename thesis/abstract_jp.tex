\clearpage
%\chapter*{要旨} \addcontentsline{toc}{chapter}{要旨}
{\huge \bf 要旨} \\

2015年、ブラックホール連星合体からの重力波GW150914をLIGOの2台の検出器が直接検出することに成功した。また2017年にはVirgoを加えた3台の重力波検出器で連星中性子合体からの重力波GW170817を検出し、さらにフォローアップ観測によって電磁波対応天体も同定され、マルチメッセンジャー観測が確立された。そして2020年にはKAGRAもLIGOとVirgoの重力波観測ネットワークに加わることで、より多くの重力波イベントの観測が期待される。

このように世界同時観測が必須な重力波検出器であるが、現在稼働している干渉計型重力波検出器のDutyCycleは $60\,\%$ 程度である。これは悪天候時の高浪や遠地でおきた地震などによる地面振動によって、干渉計の腕が変動し、干渉計が干渉しなくなるためである。これら地面振動はおよそ $1\, \mathrm{Hz}$ 以下で基線長を揺らすが、現状の防振装置ではこのような低周波地面振動は防振できない。その原因は、制御に用いている慣性センサーの低周波感度がないことによる。

本論文では、基線長補償システムについて書かれている。このシステムは、レーザーひずみ計と呼ばれる地殻変動計測のために開発された $1.5\,\mathrm{km}$ のレーザー干渉計をもちいて、KAGRAの基線長伸縮をモニターし、その信号で、メインのKAGRAの干渉計が揺れないように防振をする。

この論文では、地面振動が干渉計に与える影響について調べられており、そしてその影響を低減するための基線長補償システムの原理と、その理論的性能、既存のシステムと比較した利点が調べられている。そして、このシステムを実際にKAGRAに組み込んだ性能評価実験が述べられている。この実験では、もっとも地面振動の影響を受けやすい $3\,\mathrm{km}$ の Fabry-Perot 光共振器に基線長補償システムを組み込み、この腕共振器の長さ変動を測定した。その結果、基線長伸縮による腕共振器長変動を0.01 Hz以上では$-6 \,\mathrm{dB}$、それ以上では$-40 \,\mathrm{dB}$の低減に成功した。
