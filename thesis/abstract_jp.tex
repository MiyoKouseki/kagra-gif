%\clearpage
%\chapter*{要旨} \addcontentsline{toc}{chapter}{要旨}
%{\huge \bf 要旨} \\

2015年、LIGO の2台の重力波検出器がブラックホール連星合体イベントからの重力波 GW150914 を直接検出することに成功した。2年後の2017年には Virgo を加えた3台の重力波検出器で連星中性子合体イベントからの重力波 GW170817 を検出した。この重力波イベント直後のフォローアップ観測によって電磁波対応天体も同定され、これによりマルチメッセンジャー観測が確立された。

重力波観測には本質的に3台以上の検出器での同時観測が必要である。なぜならば単体検出器だけでは到来方向が定まらないためである。複数台で重力波を検出すれば、それらの到来時間差から到来方向が定まる。十分な到来方向を得るためには、最低でも3台の検出器で重力波を観測しなければならない。このように同時観測が必須な重力波検出器であるが、GW170817 を観測した第二次観測期間(O2)での、単独検出器の duty cycle は $60\,\%$ 程度であった。複数台の duty cycle は半分も満たない。

この duty cycle 低下の最大の原因は、おもに 1 Hz 以下の低周波地面振動である。

重力波検出器は鏡の変位測定に対して高感度であるが故に非常に狭い範囲でしか動作しない。現在の重力波検出器は、腕に km スケールの Fabry-Perot 光共振器をもつ Michelson 干渉計である。この干渉計の感度を維持させるには腕共振器を共振状態に保たなければならず、それはつまり、腕共振器長を数 nm の線幅以下に防振しなければならないことを意味する。したがって腕共振器鏡は振り子をつかって防振される。振り子をつかった受動防振は、多段にして防振比を高め、低共振周波数にしてより低周波の地面振動を防振することができる。しかし実際の振り子の共振周波数はせいぜい 1 Hz 弱であり、それ以下の地面振動は防振することができない。これに対して脈動と呼ばれる地面振動が$200\,\mathrm{Hz}$ 付近で地面を揺らす。この地面振動は波浪が海底を叩くことで生じ、その振幅は天候に強く左右され、悪天候時では数 um におよぶ。これは腕共振器の線幅よりもはるかに大きく、duty cycle の低下を直接意味する。このように、振り子を用いた受動的な防振だけでは低周波地面振動を防振することは困難である。

そこで、能動防振と呼ばれるアイデアが生まれた。このアイデアは、振り子の懸架点を支えるステージの動きをセンサーで測定し、それを打ち消すように能動的にステージを動かして防振するというものである。この制御で使われるセンサーには2つの種類がある。一つは地震計である。この地震計はステージの上に置かれ、その出力信号が小さくなるように ステージが feedback 制御され防振される。地震計はその原理から慣性センサーともよばれ、慣性系からみた地面振動を測定する。つまり地震計を用いた能動防振はステージを慣性系に対して防振することを意味する。しかし一般に地震計は低周波では感度が悪く、また tilt-holizontal coupling と呼ばれる、並進方向の地面振動が地震計の傾斜と区別がつかない信号カップリングがあり、およそ $100\,\mathrm{Hz}$ 以下では制御に使うことができない。一方でこれに対してもう一つは、suspension point interferometer (SPI) と呼ばれる干渉計である。SPIは腕共振器鏡を防振する振り子の懸架点間の距離を直接測ることができる。これを用いた能動防振は、地震計をつかった能動防振のように低周波で性能を制限されることがない。このような利点をもつSPIだが、km スケールの重力波検出器で SPI を構築する場合、SPI自身を干渉させるための角度制御が必要になるなどの技術的な困難が多く、実際の能動防振には地震計方式が用いられている。しかしやはりこの方式では、脈動は防振することができるが、100 Hz 以下の地面振動は防振できないままである。とくに、100 mHz から数 10 $\mathrm{mHz}$までの帯域では、比較的規模の大きい地震が地面を励起する。このような規模が大きい地震の場合、揺れの継続時間は数時間にもおよぶ。実際、マグニチュード6以上の地震が原因で干渉計が動作できないロックロス状態を引き起こすことが報告されており、O2での duty cycle はこの揺れによって制限されていた。

本研究では、geophysics interferometer (GIF) と呼ばれる 1.5 km のレーザーひずみ計を、 SPI として用いる能動防振の開発をおこなった。このGIF は、2つある KAGRA の 3 km の腕のうち、片腕の X アームに併設されている。このひずみ計は、地震研究所が地殻変動を精密に計測するために開発したものであり、地殻変動を直接測るために、その鏡は岩盤に固定されている。また鏡にはコーナーキューブを使用しているため、 km スケールの角度制御を必要としない。GIFは非常に安定して稼働しており、稼働を開始した2016年秋からおよそ3年間地殻変動の観測をおこなっている。このように基線長変動を測定できる GIF の信号をつかって、腕共振器を懸架するステージを feedforward 制御し、基線長が一定になるように防振をする基線長補償システムを構築した。

本論文では、地面振動が重力波検出器に与える影響について調べられており、そしてその影響を低減するための基線長補償システムの原理と、その理論的性能、既存のシステムと比較した利点が述べられている。また、このシステムを実際のKAGRAに組み込んだ性能評価実験が述べられている。この実験では、 GIF の信号をつかって Xアームの 3 km Fabry-Perot 光共振器長を防振した際の腕共振器長変動を測定した。その結果、腕共振器長の残留振動は、 0.01 Hz 以上では半分に、それ以下の帯域では約10分の1に抑えられた。

この結果から、基線長補償システムは KAGRA の duty cycle を向上させることが期待される。また、この手法は他のLIGOとVirgoにも適用可能であり、適用された場合、同時観測の duty cycle も向上が期待される。
