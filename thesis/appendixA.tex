
\chapter{Theory of Seismic Waves}


%% CLIOでの弾性定数測定によると、P波(Longitudinal Wave)とS波(Transverse Wave)それぞれの速度は、$v_{L}=5.54\pm{0.05} \mathrm{km/sec},v_{T}=3.05\pm{0.05} \mathrm{km/sec}$であった。さらに鉱山会社から報告されている岩石の密度$\rho=2.7 \mathrm{g/cm^3}$に基づくと、表\ref{table:table_1}のように弾性定数が得られる。%\cite{竹本2003}
%% \begin{table}[h]
%%   \caption{CLIOサイトの弾性定数}
%%   \label{table:table_1}
%%   \centering
%%   \begin{tabular}{cll}
%%     \hline    
%%     弾性定数 & 計算値 & 単位  \\
%%     \hline
%%     ラメ定数 $\lambda$ & $3.27\times10^11$ & $\mathrm{dyn/cm^2}$  \\
%%     剛性率 $\mu$ & $2.51\times10^11$ & $\mathrm{dyn/cm^2}$  \\
%%     ポアソン比 $\sigma$ & $0.283$ &   \\
%%     \hline
%%   \end{tabular}
%% \end{table}


\section{Rayleigh波} % ---------------------------------------------------
(レイリー波の導出。)

\section{Depth Dependence}
(レイリー波の振幅が深さに依存していることを述べる。)
