\chapter{KAGRA}
\section{Overview}
\subsection{...}
\subsection{...}
\section{KAGRA Tunnel}
\subsection{Tunnel Design}
KAGRA tunnel is excavated in the Kamioka mine in Hida, Gifu, Japan \cite{uchiyama2014excavation}. The tunnel is consisted of two floors.
干渉計を構成するほとんどの鏡は1階に設置された防振装置で懸架されているが、腕共振器を構成する4つの鏡は1階から14mの高さにある2階から懸架されている。

The tunnel is locate under 200 m from ground surface to decrease the seismic noise effectively.


\subsection{Geological features}

Hida region to which Kamioka belongs is a ancient region in Japan island \cite{Isozaki2010new}. 
\\

The main bedrock is the geniss. 

\section{Main Interferometer}
\subsection{Overview}
KAGRA is a cryogenic intergerometric gravitational-wave detector constructed at the underground site of Kamioka mine \cite{akutsu2017construction}.


\subsection{Main Interferometer}
\subsubsection{Design}
The design of KAGRA interferometer is dual recycled Fabry-Perot Michelson interferometer \cite{aso2013interferometer}\cite{somiya2012detector}.

\section{Vibration Isolation System}
\subsection{Overview}
KAGRA has 4 types vibration isolation system. 
\subsection{Type-A Suspension System}
Type-A suspensions are developed \cite{Okutomi2019development}.

\section{Summary of the Chapter}
