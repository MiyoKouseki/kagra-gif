\chapter{Geophysics Interferometer (GIF)}
KAGRA is the only GW detector, which has a strainmeter to monitor its baseline length changes. The strainmeter is named Geophysics interferometer (GIF).

GIF is a laser interferometric strainmeter, which is developed by reserchers in Earthquake Research Institute, University of Tokyo. The purpose of the strainmeter is to observe geophysical phenomena: not only earthquakes but also Earth's free oscillations. Unlike a seismometer, the strainmeter has a sensitivity in low-frequency. Moreover, unlike the continuous GPS (CGPS) nets, which also measures a strain ($\sim\,10^{-8}$), the strainmeter has more precision ($\sim\,10^{-12}$) \cite{araya2007broadband}.

In this chapter, instruments of GIF are described. After overview of GIF in section \cref{sec:sec41}, working principles of the interferometer are described in section \cref{sec:sec42}. Optics of GIF are described in section \cref{sec:sec43}. Realtime signal aquisition system to send the strain signal to KAGRA is described in \cref{sec:sec44}

\section{Overview} \label{sec:sec41}
Geophysics interferometer (GIF) is a $1500\,\mathrm{m}$ laser strainmeter constructed parallel to X-arm baseline of KAGRA. As shown in Fig.\ref{img:img402}, GIF is an asymmetric Michelson interferometer unlike symmetric KAGRA interferometer. Moreover, mirrors of the interferometer of GIF are fixed on the ground in order to monitor the baseline length changes directly. GIF is now only installed on the X-arm, which has been observing the baseline changes for almost 3 years.
\begin{figure}[h]
  \centering
  \includegraphics[width=8cm]{./img_chap4/img402.png}
  \caption{Location of geophysics interferometer (GIF). Whereas KAGRA is a symmetric L-shape $3000\,\mathrm{m}$ Michelson interferometer, GIF is an asymmetric $1500\,\mathrm{m}$ Michelson interferometer. GIF is only installed along the X-arm tunnel.} \label{img:img402}
\end{figure}

\section{Working Principle} \label{sec:sec42}
As described in section \cref{sec:12}, working principle of the strain measurement of GIF is the same as the GW detectors. However, the sensitivity of GIF is limited by the laser frequency noise due to the asymmetric optical configuration.
\subsection{Asymmetric Michelson Interferometer}
\begin{figure}[h]
  \centering
  \includegraphics[width=10.0cm]{./img_chap4/img401.png}
  \caption{Schematic drawing of the GIF as an asymmetric Michelson interferometer, which has two different arm length, $l_x\gg{l_y}$. In this figure, the mode matching optics and the optics for signal detection are not drawn.} \label{img:img401}
\end{figure}
A schematic optical layout of the GIF interferometer as an asymmetric Michelson interferometer is shown in Fig. \ref{img:img401}. The asymmetric interferometer measure change of baseline length $l_x$ with reference to the short arm $l_y$, and its fringe signal is obtained at the REFL port in the case of the GIF.

Here, we consider how the asymmetric arms affect to the optical phase of the interferometer. The relation between of the optical phase $\phi_{-}$ and the differential of the arms length ${L_{-}}=l_{\mathrm{x}}-l_{\mathrm{y}}$ is given as ${\phi}_{-} = 4\pi\frac{{L_{-}}}{\lambda}$, where $\lambda$ is the wavelength of the laser. This relation introduce the relation of the infinitesimal changes between in these phsical parameters;
\begin{eqnarray}  
  \left|\Delta \phi_{-}\right| = \frac{4\pi{L_{-}}}{\lambda}\left( \left|\frac{\Delta L_{-}}{L_{-}}\right| + \left|\frac{\Delta f}{f}\right|\right), \label{eq:eq400}
\end{eqnarray}
where $\Delta$ denote the infinitesimal change of the parameters and $f$ if the frequency of the laser, and relation $|\frac{\Delta{\lambda}}{\lambda}| = |\frac{\Delta{f}}{f}|$ was used to represent with the frequency fluctuation. Assuming enough asymmetricity of each arm length $l_{\mathrm{x}} \gg l_{\mathrm{y}}$ and the short reference arm is the rigid bar $\Delta l_{\mathrm{y}} \ll 1$ (this assumption is true because the short arm of $l_y$ is made of the super-invar plate whose coefficient of thermal expansion is extremely low), Eq.(\ref{eq:eq400}) can be represented as
\begin{eqnarray}  
  \left|\Delta \phi_{-}\right| = \frac{4\pi{l_{\mathrm{x}}}}{\lambda}\left( \left|h\right|  + \left|\frac{\Delta f}{f}\right|\right), \label{eq:eq400_a}
\end{eqnarray}
where $h = \Delta{l_{x}}/l_x$ is the strain of the baseline. It is clear that the strain and the laser frequency fluctuation are the same response to the optical phase. In other words, the frequency noise directly affects to noise of the strain measurement.

\newpage
\subsection{Seismic Strain Response}
\begin{figure}[h]
  \centering
  \includegraphics[width=10cm]{./img_chap4/img430.png}
  \caption{} \label{img:img430}
\end{figure}
Here, we consider the response from strain to the optical phase in the case that the plane seismic waves whose displacement $u(t,x)$ is represented as $u(t,x)=u_0e^{i(\omega{t}-kx)}$ with angular frequency of $\omega$ and the wavenumber of $k$. The seismic wave propagates along with the direction of the baseline of the strainmeter (right direction in this figure).

\subsubsection{Response from $u$ to $\Delta{L}$, ($H_{\mathrm{disp}}$)}
Before calculating the strain response, we calculate the response from the displacement of the seismic wave to the baseline length change. First, because the length fluctuation between two mirrors sparated with $L$ can be expressed as 
\begin{eqnarray} 
  \Delta{L(t)} &\equiv& u(t,0) - u(t,L) \\
  &=& u(t,0) - u(t-\tau,0), \label{eq:eq403}
\end{eqnarray}
where $\tau=L/v$ is the time delay, the transfer function from the displacement to the length fluctuation is given by Laplace transform as
\begin{eqnarray} \label{eq:eq404}
  H_{\mathrm{disp}}(s) \equiv \frac{\Delta{L(s)}}{u(s)} = \frac{u(s)\left[ 1-\exp(-\tau{s}) \right]}{u(s)} = 1 - \exp(-\tau{s})
\end{eqnarray}

\subsubsection{Response from $\epsilon$ to $\phi_{-}$, ($H_{\mathrm{strain}}$)}
Because the strain amplitude $\epsilon{(x,t)}$ is defined as $\epsilon{(x,t)}\equiv\frac{du}{dx}$, the seismic strain is represented as 
\begin{eqnarray} 
  \epsilon{(x,t)} \equiv \frac{du}{dx} = \frac{du}{dt} \frac{1}{v} = \frac{s}{v}u(s) \label{eq:eq406}
\end{eqnarray}
Therefore, the transfer function from the seismic strain to the displacement is given  as
\begin{eqnarray} \label{eq:eq407}
  \frac{\Delta{L(s)}}{\epsilon(s)} = H_{\mathrm{disp}} \frac{v}{s}
\end{eqnarray}
Finally, because the transfer function from the length change of the baseline to the optical phase is given as $4\pi/{\lambda_{\mathrm{opt}}}$, the transfer function from the seismic strain to the optical phase is represented as 
\begin{eqnarray} \label{eq:eq407}
  H_{\mathrm{strain}}(s) = 4\pi\frac{1}{\lambda_{\mathrm{opt}}} \left[1 - \exp(-\tau{s}) \right]\frac{v}{s}.
\end{eqnarray}

Here, as a summary of these transfer function, these are related with each other as shown in Fig.(\ref{img:img411}). 

\begin{figure}[h]
  \centering
  \includegraphics[width=10.0cm]{./img_chap4/img411.png}
  \caption{The response from seismic strain to optical phase.} \label{img:img411}
\end{figure}

\subsubsection{Improvement of the sensitivity with longer baseline}
Here, we describe the length dependance of the strain response given by Eq.(\ref{eq:eq407}). Bode plot of the strain response with two different baseline length is shown in Fig. \ref{img:img411_a}, in the case that the phase velocity is $5.5\,\mathrm{km}$. One can find that the DC gain is greater for $L=3000\,\mathrm{m}$ than the gain for $L=3000\,\mathrm{m}$, and the corner frequency is lower in the case of long baseline.

Because the corner frequency $f_0\equiv {1}/{\tau}$ is given as
\begin{eqnarray}
  f_0 = \frac{v}{L},
\end{eqnarray}
if the baseline length is twice, the frequency is half, which means decrease of the observation frequency band. For example, in the case of $L=1500\,\mathrm{m}$, and assuming the phase velocity of $5.5 \mathrm{km/sec}$, the corner frequency is $f_0\sim3.7\,\mathrm{Hz}$. Below this frequency, therefore, the GIF interferometer responses the strain as the flat response.

\begin{figure}[p]
  \begin{center}
    \includegraphics[width=13.0cm]{./img_chap4/img412.png}
    \caption{Compasison of the transfer function from strain of the baseline $\epsilon$ to the length change of that $\Delta{L}$ in the different baseline length. $3000\,\mathrm{m}$ の基線長ではその半分の$1500\,\mathrm{Hz}$よりも、DCゲインは二倍大きい一方で、コーナー周波数は \color{red}{A} になり帯域が減る。また、周波数が \color{red}{B} の条件を満たすとき、ゲインはゼロになる。なぜならば、このときひずみは基線を同相で動かし、基線長伸縮として現れないためである。}\label{img:img411_a}
  \end{center}
\end{figure}

\subsection{Noise}
\subsubsection{Frequency Noise}
先述したように、GIFのような1500mと70cmの腕を持つ非対称マイケルソン干渉計は、腕の同相雑音除去が効かない。周波数ノイズは
\begin{eqnarray}
  h = \frac{\Delta{f}}{f} \sim 7\times10^{-13} [\mathrm{1/\mathrm{Hz}}]
\end{eqnarray}
になる\cite{araya2017design}。

\subsubsection{Residual Gas Noise}
Bcause residual gas fluctuates the optical path, length measured by interferometer is also fluctuates. The opttical path $L_{\mathrm{opt}}$ is given by $L_{\mathrm{opt}}=nL$, where $L$ is the length of the baseline and $n$ is the refraction index in the optical path relative to the path in the vacuum. Under the pressure of $p$ in vacuum, the index $n$ is approximated as $n = 1 + c_0(p/p_0)$, where $c_0$ denotes the relattive refractive index, $p_0$ is pressure in standard air at 1 atm. The apparent strain due to the residual pressure is given as \cite{ciddor1996refractive};
\begin{eqnarray}
  h = (L_{\mathrm{opt}}-L)/L = c_0(p/p_0) \sim 3\times10^{-9} p.
\end{eqnarray}
In order to maintain the strain sensitivity; $3\times10^{-13}$, the vacuum pressure should be below $1\times10^{-4}\,[\mathrm{Pa}]$. However, actual vacuum pressure is $1\times10^{-2}\,[\mathrm{Pa}]$, then strain is $\sim\times10^{-12}$.



\section{Optics} \label{sec:sec43} %光学系
これまでの議論はレーザー光を平面波として扱っていた。しかしながら実際のレーザー光は伝搬する距離で位相のずれやビームサイズの変化が生じる。このようなビームを有限の範囲でかんしょうさせるにはこれらビームプロファイルを適切に設計しなければならない。ここでは、レーザー光がガウシアンビームだとして、$1500\mathrm{m}$の非対称マイケルソン干渉計を干渉させるために必要なモードマッチについて述べる。

\subsection{Gaussian Beam}
理想的なレーザー光は$\mathrm{TEM}_{00}$と呼ばれる空間モードをもち、電場の位相は距離に応じて変化する。この空間モードをもつビームのことをガウシアンビームと呼ぶ。このガウシアンビームが$z$軸に伝搬する場合を考える。この電場は
\begin{eqnarray}
  u(x, y, z)=\sqrt{\frac{2}{\pi{w^2(z)}}} \exp \left(i\zeta(z)-\mathrm{i} k \frac{x^{2} +y^{2}}{2 R(z)}-i\frac{2\pi}{\lambda}z\right)
  \exp \left(-\frac{x^{2}+y^{2}}{w^{2}(z)}\right)  \label{eq:eq415}
\end{eqnarray}
とかける\cite{bond2016interferometer,svelto1998principles}。ここで、$\lambda,\,w_0$はそれぞれレーザーの波長、$z=0$でのビーム径である。また
\begin{eqnarray}
  z_0 &=& \frac{\pi{w^2_0}}{\lambda} \\ \label{eq:eq415_a}
  w(z) &=& w_0\sqrt{1+\left(\frac{z}{z_0}\right)^2}, \\ \label{eq:eq415_b}
  R(z) &=& z\left[1+\left(\frac{z_0}{z}\right)^2\right],\\ \label{eq:eq415_c}
  \phi(z) &=& \arctan\left(\frac{z}{z_0}\right) \label{eq:eq415_d}
\end{eqnarray}
はそれぞれ、Reyliegh length、$z$でのビーム径、曲率、Gouy位相である。このときEq.(\ref{eq:415})から、Fig.\ref{img:img415a}にしめすように、ガウシアンビームのパワー$P=|u^2|$はガウス分布をもつことがわかる。さらにビーム径はビーム強度が$1/e^2$になるときの半径とわかる。
\begin{figure}[p]
  \begin{minipage}{14cm}
    \centering    
    \includegraphics[width=8cm]{./img_chap4/img415a.png}
    \subcaption{Evolution of a Gaussian beam propagating along the z-axis\cite{riehle2006frequency}}{$w_0$ denotes a beam radius at beam weist, where $z=0$. $w(z)$ and $R(z)$ are the beam radius and curvature at $z$. Gouy phase is not shown in here.}\label{img:img415a}
  \end{minipage}\\
  \begin{minipage}{14cm}
    \centering        
    \includegraphics[width=14cm]{./img_chap4/img415.png}
    \subcaption{Beam prifile}{(left) Beam radius normalized by $w_0$ as a function of $z/z_0$, where $z_0$ is Rayleigh length. (Middle) Beam curvature normalized by $z_0$. (right) Gouy phase.}\label{img:img415}    
  \end{minipage}
  \caption{Gaussian beam.}
\end{figure}


ガウシアンビームを特徴づける$z$の関数であるパラメータEq.(\ref{eq:eq415_b,eq:eq415_c,eq:eq415_d})をFig.{\ref{img:img415}}に示す。$z=0$のとき、ビーム径は最も小さくEq.(\ref{eq:eq415})の位相は0であるため、ガウシアンビームは平面波とみなせる。一方で、$z\gg{z_0}$のときレーザー光源は点光源とみなせ、球面波としてふるまう。


\subsection{Reflector Design}
リフレクタの大きさを最小限にするために、GIFの干渉計はエンドミラーでビーム直径が最も小さくなるビームウエストがくるようにしている。この場合、ビームウエスト$w_0$を小さくしたいが、小さくしすぎると$L=1500\,\mathrm{m}$離れたBSとフロントリフレクタで大きくなるので、できるだけフロントでのビーム径$w(L)$はエンドのビーム径$w_0$に対してできる限り小さくしたい。つまりこれを式で表すと、
\begin{eqnarray}
  \argmin_{w_0} \left[w_0\times\frac{w(L)}{w_0}\right] \label{eq:eq415_e}
\end{eqnarray}
となるような$w_0$を探せばよい。Eq.(\ref{eq:eq415_b})をEq.(\ref{eq:eq415_e})に代入して解けば
\begin{eqnarray}
  w_0 = \sqrt{\frac{{L\lambda}}{\pi}}
\end{eqnarray}
を得る。つまりビームウエストサイズ$w_0=\sqrt{{1500\,\mathrm{[m]}}\times 532\,\,\mathrm{[nm]}/\pi} = 16\,\mathrm{mm}$となる。このときのフロントリフレクタでのビーム径は$w(L)=\sqrt{2}{w_0}$になる。ちなみに、リフレクタの大きさはフロントリフレクタでのビーム径の3倍の大きさを往復できるようにするには、最低限必要なリフレクタの aperture diameter は$2\times3\times\sqrt{2}w_0\sim270\,\mathrm{mm}$となる。

\subsection{Input Output Optics}
レーザー光源から出射されたビームを適当な大きさにして干渉計へ入射するために、input output optics と呼ばれる光学系を構築している。Fig(\ref{img:img416})にGIFのinput output optics と干渉計を示す。光源からの出射ビームは、エンドリフレクタの位置Aでビームウエストになるように、コリメータ(1)とステアリングミラー(2)、凹面鏡(3)を経てBSへと入射される。2つのリフレクタから反射してきたビームは地点Bで再結合し、2つ目の凹面鏡とコリメータ(4)をへてPDに入射する。これらopticsの調整をおこない干渉信号を得ている\cite{miyo2017baseline}。

\begin{figure}[h]
  \begin{center}   
    \includegraphics[width=14cm]{./img_chap4/img416.png}
    \caption{Schematic optics layout}{(1) A collimator lens for input beam. (2) A flat mirror for steering mirror. (3) Two concave mirrors with a radius of curvature of $9.8\,\mathrm{m}$ for mode matching. (4) A collimator lens for output beam. The waist of the beam is at the end reflector at point A. Two reflected on the reflectors are combined at point B.}\label{img:img416}
  \end{center}
\end{figure}


\subsection{Core Optics}
The core optics of the Michelson interferometer are composed of two reflectors and beam splitter (BS). 

\begin{figure}[h]
  \begin{minipage}[b]{7cm}
    \begin{center}   
      %\includegraphics[width=7cm]{./img_chap4/img418.png} % ファイル重い
      \includegraphics[width=7cm]{./img.png}
      \subcaption{Core optics in the front vacuum chamber. }\label{img:img418}
    \end{center}
  \end{minipage}\hspace{0.1cm}
  \begin{minipage}[b]{7cm}
    \begin{center}   
      %\includegraphics[width=7cm]{./img_chap4/img419.png} %ファイル重い
      \includegraphics[width=7cm]{./img.png}      
      \subcaption{Core optics in the end vacuum chamber. }\label{img:img419}
    \end{center}
  \end{minipage}
  \caption{}  
\end{figure}


\subsection{Frequency Stabilized Laser}
Because GIF is an asymmetric Michelson interferometer, the frequency stability of the laser would limit the sensitivity of the strain, and we use the frequency stabilized laser, which is stabilized the laser frequency to the iodine absorption line \cite{araya2002iodine}. The control diagram of the frequency stabilization system is shown in Fig.\ref{img:img417}. このシステムはヨウ素分子の吸収スペクトル線の周波数とレーザーの周波数との差を利用したフィードバック制御である。エラー信号は、ポンプ光とプローブ光をつかったドップラーフリーな吸収線信号\cite{snyder1980high}をPDH法をつかって取得する。

\begin{figure}[h]
  \begin{center}   
    \includegraphics[width=12cm]{./img_chap4/img417.png}
    \caption{Schematic diagram of the frequency-stabilization system of the GIF main laser.}\label{img:img417}
  \end{center}
\end{figure}

\section{Realtime Signal  Aquisition System} \label{sec:sec44}



\subsection{Quadrature Phase Fringe Detection}
\begin{figure}[h]
  \begin{center}
    \includegraphics[width=13.0cm]{./img_chap4/img413.png}
    \caption{Quadrature interferometer used in the GIF strainmeter. A half-wave plate (HWP) produces a p-polarization and s-polarization. A quator-wave plate (QWP) delay the optical phase of the s-polarized light with 90 degree against to the another. As a result, one can obtain the quadrature phase fringe.}\label{img:img413}
  \end{center}
\end{figure}

We use the quadrature phase fringe detection to measure the length change of the baseline with wide dynamic range \cite{bobroff1993recent}. The optical layout for the detection is shown in Fig.(\ref{img:img413}).

The quadrature phase fringes are detected by two photo detectors, these can be represented as
\begin{eqnarray}
  x(t) &=& x_0 + a \sin(\phi(t)+\phi_0), \\
  y(t) &=& y_0 + b \cos(\phi(t)),
\end{eqnarray}
where $x$ and $y$ are the two voltage outputs from the detectors, $a$ and $b$ are the amplitudes of these fringe signals, $x_0$ and $y_0$ are the offsets, $\phi$ is optical phase, and $\phi_0$ is the phase offsets from imperfections \cite{zumberge2004resolving}.
このとき、位相角$\phi$は
\begin{eqnarray}
  \tan{\phi(t)} = \frac{1}{{\cos(\phi_0)}} \left(\displaystyle{\frac{b}{a}\frac{x(t)-x_0}{y(t)-y_0}-\sin(\phi_0)}\right)
\end{eqnarray}
で表される。つまりある時刻$t$のときに、パラメーター$x_0,\,y_0,\,a,\,b,\,\phi_0$が与えられれば、そのときの位相$\phi(t)$は求まる。

\subsection{Realtime Data Processing}
KAGRAのデジタルシステムをつかってリアルタイムで楕円パラメータを取得する。KAGRAのデジタルシステムでは

\cite{bork2001overview}


GIFからの2つの干渉信号を

Fig.\ref{img:img420}にひずみ変換のMatlabのSimlinkモデルを示す。

\begin{figure}[h]
  \centering
  \includegraphics[width=15.0cm]{./img_chap4/img420.png}
  \caption{}\label{img:img420}
\end{figure}



\section{Summary of the Chapter} %章のまとめ
本章で述べたパラメータを表にまとめる。
