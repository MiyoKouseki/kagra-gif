\chapter{Demonstration of Baseline Compensation}
To demonstrate the baseline compensation system, we used X-arm cavity.

\section{Experimental Arrangement}
Because the purpose of the baseline compensation system is to reduce the arm cavity length fluctuation, we prepared the experimental arrangement to measure the length.

\subsection{Measurement of X-arm cavity length}
The length fluctuation was measured by the frequency change of the main infrared laser when the cavity is on resonance by feedback control using the acousto-optics modulator (AOM). Brief measurement procedure is shown in Figure \ref{img:img600}. (1) The deformation of the baseline cause the length change of the arm cavity length through the suspensions. Suppose that the baseline length is displaced by $\Delta{L}$ from the nominal length
of $L$. Utilizing te PDH method, we can obtain the signal proportional to this displacement. (2) This signal is also interpreted as the frequency changes of the input laser because the frequency change $\Delta{f}$ has a relation with the baseline length change $\Delta{L}$;
\begin{eqnarray}
  \displaystyle -\frac{\Delta{f}}{f} = \frac{\Delta{L}}{L}.
\end{eqnarray}
(3) To keep the optical cavity on resonance, the signal is fed back to the AOM, which is the frequency actuator. Therefore we can measure the cavity length with the feedback signal to the AOM.
\begin{figure}[h]
  \centering
  \includegraphics[width=13cm]{./img_chap6/img600.png}\label{img:img600}
  \caption{Experimental arrangement for X-arm length measurement. X-arm cavity controled by feeding the PDH signal back to the AOM of the input laser to keep on resonance. The length change of the cavity is obtained from the feedback signal.}
\end{figure}


\subsection{Control Design}
\begin{figure}[h]
  \begin{minipage}{14cm}
    \begin{center}   
      \includegraphics[width=14cm]{./img_chap6/img630a.png}
      \subcaption{Schematic contol of each platform stage. Left figure is that of the IX stage, right figure is that of the EX.}\label{img:img630a} \hfill\vspace{10pt}
    \end{center}
  \end{minipage}
  \begin{minipage}{14cm}
    \begin{center}   
      \includegraphics[width=14cm]{./img_chap6/img630b.png}
      \subcaption{Control block diagram of each platform stage. Left figure is that of the IX stage, right figure is that of the EX.}\label{img:img630b}
    \end{center}
  \end{minipage}
  \caption{The baseline compensation control of each platform stage for demonstration.}
\end{figure}

To demonstrate the baseline compensation system using GIF, we design the simple control configuration. Although the most simple configuration is only the feedforward using the GIF, only the feedforward control cannot suppress the disturbance other than the horizontal seismic noise such as the tilt ground motion and the temperature fluctuation \cite{sekiguchi2016astudy}. Because these disturbance could move the pratform stage in horizontal direction, we need a feedback control using the position sensor to supress these disturbances. Therefore we use the sensor correction control rather than the feedfoward control.

Figure \ref{img:img630a} shows the schematic control of the pratform stage for the input x-arm test mass (IX) and end x-arm test mass (EX). While the IX stage is fed back the relative postion sensor singal to the actuator on the stage, the EX stage is added the GIF stainmeter signal. In other words, while the IX stage is locked to the local IX ground, the EX stage is also locked to the local EX ground but this feedback signal is corrected by using the GIF strainmeter. The GIF measure the baseline length changes, which means the differential motion of the IX and EX gorund. Therefore, the feedback signal corrected by using GIF is the same as the feedback signal of the IX stage. Thus, the EX stage can follow the IX stage by using the corrected feedback signal.

Figure \ref{img:img630b} shows the control diagram of each stage. In both stages, the displacement of the IX pratform stage $X_{\mathrm{STG}}$ is disturbed by the local seismic motion $X_{\mathrm{GND}}$ though the mechanical response of the inverted pendulum (IP) $H_{\mathrm{s}}$. Moreover the displacement of the IX testmass is also disturbed by this seismic noise through the mechanical response of the pendulum $H_{\mathrm{TM}}$. To reduce the test mass motion in low-frequency region, below 1 Hz, the platform stage is controled by the feedback control using the relative position sensor. $S_{\mathrm{L}},\,N_{\mathrm{fb}}$ and $B_{\mathrm{L}}$ are the displacement response and the noise of the relative position sensor and the low-pass filter not to inject the sensor noise to the feedback signal. The feedback signal is sent to the actuator, whose transfer function from the actuator force to the platform stage is given by $P_{\mathrm{a}}$, through the control filter $C_{\mathrm{fb}}$. On the other hand, the feedback signal of the EX stage is corrected by the GIF signal.

In this situation, the each displacement of the stage are given by 
\begin{eqnarray}
  X_{\mathrm{STG(IX)}} &=& \displaystyle\frac{G}{1+G} X_{\mathrm{GND(IX)}} + \frac{G}{1+G} N_{L} + \frac{1}{1+G} H_{\mathrm{s}} X_{\mathrm{GND(IX)}} \\ \nonumber,
  X_{\mathrm{STG(EX)}} &=& \displaystyle\frac{G}{1+G} \left(1- \frac{C_{\mathrm{sc}}S_{\mathrm{wit}}}{B_{\mathrm{L}}S_{\mathrm{L}}}\right) X_{\mathrm{GND(EX)}} + \frac{G}{1+G}N_{L} \\ \nonumber
  &+& \frac{G}{1+G} \frac{C_{\mathrm{sc}}S_{\mathrm{wit}}} {B_{\mathrm{L}}S_{\mathrm{L}}} X_{\mathrm{GND(IX)}}
  + \frac{G}{1+G} \frac{C_{\mathrm{sc}}S_{\mathrm{wit}}} {B_{\mathrm{L}}S_{\mathrm{L}}} N_{\mathrm{wit}}\\ 
  &+& \frac{1}{1+G} H_{\mathrm{s}} X_{\mathrm{GND(EX)}},
\end{eqnarray}
respectively, where $G=C_{\mathrm{fb}}P_{\mathrm{a}}S_{\mathrm{L}}B_{\mathrm{L}}$ is the loop gain. Here, if $G\gg1$ and we design the sensor correction filter $C_{\mathrm{sc}}$ so that
\begin{eqnarray}
  \frac{C_{\mathrm{sc}}S_{\mathrm{wit}}}{B_{\mathrm{L}}S_{\mathrm{L}}} = 1,
\end{eqnarray}
the displacement of each stage are give as 
\begin{eqnarray}
  X_{\mathrm{STG(IX)}} &=& X_{\mathrm{GND(IX)}} + N_{\mathrm{L}},\\
  X_{\mathrm{STG(EX)}} &=& X_{\mathrm{GND(IX)}} + N_{\mathrm{L}} + N_{\mathrm{wit}}.
\end{eqnarray}
Moreover, if the noise of the GIF, which is the wittness sensor is smaller than that of the relative position sensor, both stage motions are the same each other; $X_{\mathrm{STG(EX)}}=X_{\mathrm{STG(IX)}}$. This same motion means the reduction of the differential stage motion. Thus, the cavity length is isolated from the differential ground motion, which is the baseline length fluctuation.




\section{Results}

\subsection{Demonstrasion}

\subsection{Results}
\begin{figure}[h]
  \centering
  \includegraphics[width=12cm]{./img_chap6/img610.png}
  \caption{Length change of both X-arm baseline and X-arm cavity when baseline compensation system is turned on or off. At 12 minutes, the control is on.}\label{img:img610}
\end{figure}

Figure \ref{img:img610} shows the length fluctuation of the arm cavity and of the baseline as a reference. At 12 minutes, the baseline compensation system was turned on. Whereas the X-arm cavity length is drifted duaring the compensation system was off, the drift is removed duaring the system was on. This drift is comparable to the earth tide. As a result, this system compensated the deformation of the baseline, and reduction ratio is almost 1/10.

This result also indicate that the RMS amplitude of the X-arm cavity length is reduced. The amplitude spectrum density of the length when both the compensation system was on and off is shown in Figure \ref{img:img611}. It is crear that the acumurated RMS amplitude is reduced due to the compensation system. In the next, we compare this measured data with the rigid body model.

\begin{figure}[h]
  \centering
  \includegraphics[width=10cm]{./img_chap6/img611.png}
  \caption{ASDs of X-arm caivty length when baseline compensation system is turned on and off. }\label{img:img611}
\end{figure}

\subsection{Comparison with the model}
Compare with the measured data and rigid body model of the KAGRA suspensions, which have been developed \cite{sekiguchi2016astudy}. Bcause this model outputs the state space model, we can calculate the transfer function. Supposing the CMRR is large enough to ignore the coupling from the common motion to the differential motion, as described in \cref{sec:}, the transfer function from the...\textcolor{red}{AAAAAA}


このモデルは、剛体モデルに基づいて力学の状態空間モデルを計算する\cite{sekiguchi2016astudy}。この状態空間モデルをつかって、地面振動からテストマスまでの伝達関数等を計算した。ただし、この伝達関数は振り子単体の伝達関数なので、ふたつの防振装置で防振される腕共振器を計算する場合、\cref{sec}で述べたように、CMRが十分大きい場合を仮定する。つまり、基線の地面振動の同相成分から腕共振器長変へのカップリングは無視できるものとする。

\subsubsection{When the compensation system is OFF}
補償システムがOFFのときのXアーム長変動の測定値と、そのときにモデルから期待されるものを比較する。Fig.\ref{img:img612}にその比較を示す。\cref{sec}で述べたように、補償システムを入れる前は、それぞれのPre-isolatorはそれぞれのローカルのLVDTをつかってFeedback制御されている。このときのLVDTのセンサーノイズを水色で、地面振動ノイズをオレンジ色で示す。これらノイズの二乗和のルートをTotalとし、赤色で示す。1Hz以上の、Xアーム測定の測定ノイズで埋もれている帯域を除けば、1Hz以下ではこのTotalは測定値と一致していることがわかる。

\subsubsection{When the compensation system is ON}
補償システムがONのときの比較をFig.\ref{img:img613}示す。ただし、we assumed the reduction factor of the sensor correction of 1./20 as mentioned in \cref{sec}. 補償システムがOFFになっていたときの比較では、1Hz以下でTotalと測定値が一致していたが、ONの場合では、予測されるTotalと測定値は一致していない。





\begin{figure}[p]
  \begin{minipage}{15cm}
    \begin{center}   
      \includegraphics[width=9cm]{./img_chap6/img612.png}
      \subcaption{Noise budget of the X-arm length fluctuation when the compensation system is OFF. Measurement is same as the black line in Fig.\ref{img:img611}. Total is the summation of all noise contributions.}\label{img:img612} \hfill\vspace{10pt}
    \end{center}
  \end{minipage}
  \begin{minipage}{15cm}
    \begin{center}   
      \includegraphics[width=9cm]{./img_chap6/img613.png}
      \subcaption{Noise budget of the X-arm length fluctuation when the compensation system is ON. Measurement is same as the red line in Fig.\ref{img:img611}. Total is the summation of all noise contributions assuming the reduction factor of sensor correction of 1/20.}\label{img:img613}
    \end{center}
  \end{minipage}
  \caption{Noise budget of the X-arm length fluctuation when the compensation system if turned on or off.}
\end{figure}



\section{Discussion and Summary of the Chapter}
\subsection{Discussion}
\subsection{Summary}
