\chapter{Conculusion and Future Directions} \label{chap6}
\section{Conclusion}
The conclusion is below:
\begin{itemize}
\item The low-frequency seismic noise is a problem for the stable operation of the km-scale baseline GW detectors.
\item The baseline compensation system is developed and demonstrates the reduction of the cavity length fluctuation below 1 Hz.
\item This baseline compensation system is an important role to improve the duty cycle of the current and future GW detectors
\end{itemize}


\section{Future Directions}
In order to improve the isolation performance from 0.1 to 1 Hz where our compensation system could not reduce effectively, of course, the internal DOFs coupling should be resolved firstly, but the active inertial seismic isolation is needed. Through the study of the baseline compensation system, we obtained some prospects for improving the seismic isolation system. The prospects are below:
\begin{itemize}
\item Above 1 Hz, the observation frequency band for the GW detector, the passive vibration isolation system using the multi-stage pendulum, should be used. 
\item From 0.1 Hz to 1 Hz, the eigenfrequencies of the pendulums, the active inertial seismic isolation system using the inertial sensor should be used, because this system can suppress both common and differential motion of the arm cavity. This advantage can resolve the problem that the active baseline isolation system has due to the CMRR of the cavity's mechanical response.
\item Below 0.1, the frequency where the sensitivity of the inertial sensor is worse, the active baseline seismic isolation system using the GIF strainmeter should be used.
\end{itemize}
This seismic isolation system optimized in these three frequency regions will improve the operation stability of current and future GW detectors.
