%\documentclass[a4paper,twoside,12pt]{book}
\documentclass[a4paper,twoside,12pt,openright]{book}
\usepackage[DIV=10,BCOR=3mm,headinclude=true,footinclude=false]{typearea}
%
\usepackage{jtygm}
\usepackage{amsmath,amssymb}
\usepackage{bm}
\usepackage[dvipdfmx]{graphicx,color,hyperref,rotating}
\usepackage{capt-of}
%\usepackage[a4paper,width=150mm,top=25mm,bottom=25mm]{geometry}
\usepackage[hang,small,bf]{caption}
\usepackage[subrefformat=parens]{subcaption}
\captionsetup{compatibility=false}
\usepackage{here}
\renewcommand{\baselinestretch}{1.2}
\hypersetup{
  colorlinks=true,
  linkcolor=blue,
  bookmarksnumbered=true,
  pdfborder={0 0 0},
  bookmarkstype=toc
}
% cref
\usepackage{cleveref}
\crefformat{section}{#2#1#3} % see manual of cleveref, section 8.2.1
\crefformat{subsection}{#2#1#3}
\crefformat{subsubsection}{#2#1#3}

%
\title{{\LARGE \sc Thesis}\\ \vspace{2zh}{\bf \LARGE A Study of Baseline Compensation System for Stable Operation of Gravitational-wave Telescope }\\ \vspace{8zh}}
\author{{\LARGE \bf Koseki Miyo} \\ \\ {\LARGE  \it Department of Physics} \\ {\LARGE \it University of Tokyo}}
\date{\vspace{1zh} \LARGE MMM 2020}
%
%
\begin{document}
\setcounter{tocdepth}{2}
\maketitle
\tableofcontents
\chapter{Background}
本章では、重力波とその検出器について述べる。


\section{Gravitational-wave}
重力波は光速で伝わる時空の歪みである。この歪みは一般相対性理論の結果として導き出されることがアインシュタインによって1919年に示された。しかしながらその歪は非常に小さく、およそ100年後のLIGOによる初検出まで直接検出はされなかった。

\subsection{Properties of GWs}
\subsubsection{Two polized transverse wave}
The interval between two events in space-time is
described with the metric tensor $g_{\mu\nu}$ as, 
\begin{eqnarray}
  d s^{2}=g_{\mu \nu} d x^{\mu} d x^{\nu} (\mu,\nu = 0,1,2,3),
\end{eqnarray}
where $dx^{\mu}$ represents the coordinate distance of the events, and $x^{\mu}$ has 4 components; $(ct,x,y,z)$.

In the general relativity theory\cite{einstein1916vd}, the metric tensor $g_{\mu\nu}$ is described by the Einstein's equation;
\begin{eqnarray}
  R_{\mu \nu}\left(g_{\mu \nu}\right)-\frac{1}{2} g_{\mu \nu} R\left(g_{\mu \nu}\right)=\frac{8 \pi G}{c^{4}} T_{\mu \nu},
\end{eqnarray}
where $R_{\mu\nu}$ is the Ricci tensor, $R=g^{\mu \nu} R_{\mu \nu}$ is the Ricci scalar curvature, $T_{\mu\nu}$ is the energy-momentum tensor, $G$ is the Newton's gravitational constan, and $c$ is the speed of light.

GW is derived from this Einstein's equation when the metric can be described as the perturbation $h_{\mu\nu}$ to the Minkowsky space-time $\eta_{\mu\nu}$, thus
\begin{eqnarray}
  g_{\mu \nu}=\eta_{\mu \nu}+h_{\mu \nu}.
\end{eqnarray}
In this weak-field regime, the Einsteins's equation is reduced to a linearized wave-equation whose solution is represented as
\begin{eqnarray}
  h_{\mu \nu}(z, t)=\left(\begin{array}{cccc}{0} & {0} & {0} & {0} \\ {0} & {-h_{+}} & {h_{\times}} & {0} \\ {0} & {h_{\times}} & {h_{+}} & {0} \\ {0} & {0} & {0} & {0}\end{array}\right) \cos \left[\omega\left(\mathrm{t}-\frac{\mathrm{Z}}{\mathrm{c}}\right)\right],
\end{eqnarray}
where $\omega$ is the angluar frequency of GW, $z$ is the propagation direction of the wave, $h_{+} \text {and } h_{\times}$ are the independent polization of that. Therefore, GW is the transverse wave propagating with speed of light.

The two polization of GW are known as plus and cross polization, and these polization change the distance between two points as shown in Fig.\ref{img:img131}. 

\begin{figure}[h]
  \begin{center}   
    \includegraphics[width=11.0cm]{./img_chap1/img131.png}
    \caption[Polization of the GW]{Polization of the GW propagating in the direction of the paper. These polization change the distance as the tidal motion.
}\label{img:img131}
  \end{center}
\end{figure}

\subsubsection{Radiation}
The wave amplitude $h_{\mu\nu}$ is proportional to the second time derivative of the quadrupole moment of the source \cite{einstein1918gravitationswellen};
\begin{eqnarray}
  h_{i j}=\frac{2}{r} \frac{G}{c^{4}} \ddot{Q}_{i j}^{T T}\left(t-\frac{r}{c}\right),
\end{eqnarray}
where 
\begin{eqnarray}
  Q_{i j}^{T T}(x)=\int \rho\left(x^{i} x^{j}-\frac{1}{3} \delta^{i j} r^{2}\right) d^{3} x
\end{eqnarray}
is the quadrupole.

\textcolor{red}{もうすこし具体的なQuadrupoleをしめして、現実的に検出できそうな重力波の大きさを述べる。}

\subsection{Sources of Gravitational-wave}
\subsubsection{Compact Binary Coalescence}
Compact binary coalescence (CBCs), such as black holes and neutron stars, emit a characteristic chirp GW signal. The frequency of a hirp GW signal $f_{g}$ increase as a function of time. This growing up is caused by loosing the angluar momentum of the system due to the emittion of GW. 

Advanced LIGO have detected the first GWs from stellar-mass binary black holes (BBHs) in the first observation run (O1), which took place from September 12, 2015 until January 19, 2016. After this observation, Virgo detector joined the Advanced LIGO detectors and this network have detected the first detection of GWs from a binary neutron star inspiral in the second observation run (O2), which ran from November 30, 2016 to August 25, 2017. Moreover, observation of GWs from a total of seven BBHs \cite{abbott2019gwtc}.


\subsubsection{Continuous GWs}
Without rotating two objects, asymmetric spinning stars, such as neutron stars and pulsars, could produce detectable GWs which signal is also well defined \cite{leaci2012searching,hereld1984search}.

\subsubsection{Burst GWs}
In addition to continuous gravity waves, there are short suration GWs like a burst event. Supernovae are good candidates to emit te burst GWs \cite{ott2004gravitational}

\subsubsection{Stochastic GWs}
The stochastic background GWs are predicted\cite{starobinskii1979spectrum,Christensen_2018}. This background signal is originated from quantum fluctuations during inflation \cite{PhysRevD.23.347}. Basically, stochastic bacground will appear like a random noise in an individual detector. However, it will be found like a coherent signal in two detector.

%\cite{damour2005gravitationa}




\section{Interferometric Gravitational-wave detection} \label{sec:12}
Basic design of a terrestrial GW detectors are Michelson interferometer \cite{weiss1972electronically}. This interferometer sensitives to the differential length change of its arms. This change is a strain caused by GWs. We assume that puls mode of GW is passing thorugh the interferometer perpendicularly.

\subsection{Michelson Interferometer} \label{sec:121}
\begin{figure}[h]
  \begin{center}   
    \includegraphics[width=8.0cm]{./img_chap1/img132.png}
    \caption{Michelson Interferometer. }\label{img:img132}
  \end{center}
\end{figure}

Michelson interferometer is a converter from the optical phase difference of two lights to the amplitude modulation of a single light. Consider about the interferometer shown in Fig. \ref{img:img132}. Incident light can be wrtten as,
\begin{eqnarray}
  E_{\mathrm{in}} = E_{0} e^{i\omega{t}},
\end{eqnarray}
where $E_0$ is the amplitude and $\omega_0$ is the angular frequency of the laser field
. Two lights splited by the Beam Spliter (BS) interferer at the Anti-symetric (AS) port and Refrection (REFL) port. The output fieled at the AS port is represented as,
\begin{eqnarray}
  E_{\mathrm{AS}} = -\frac{1}{2}rE_{0} e^{i\left(\omega_{0} t-\phi_{x}\right)}+\frac{1}{2}r E_{0} e^{i\left(\omega_{0} t-\phi_{y}\right)},
\end{eqnarray}
where $r$ denote the amplitude reflectivity of the end mirrors, and $\phi_{x}$ and $\phi_{x}$ are the phase delay due to the light traveling in the $x$ and $y$ arms. This output signal can be represented as a single fieled as,
\begin{eqnarray}
E_{\mathrm{AS}} = i r E_{0} e^{i\left(\omega_{0} t-\left(\phi_{x}+\phi_{y}\right) / 2\right)} \sin \left(\frac{\phi_{x}-\phi_{y}}{2}\right). \label{eq:eq132}
\end{eqnarray} 
Wwe find that the amplitude of the output light is a function of the difference between two phases; $\phi_{x}-\phi_{y}$. Furthermore, the power of output light at the AS port is obtained by squaring the Eq.\ref{eq:eq132}, 
\begin{eqnarray}
  P_{\mathrm{AS}} &=\left[r\sin({\phi_{-}})\right]^2P_0  \label{eq:eq133}
\end{eqnarray}
Similarly, power of the output light as REFL port is written as,
\begin{eqnarray}
  P_{\mathrm{REFL}} &=\left[(r\cos({\phi_{-}}))\right]^2P_0. \label{eq:eq134}
\end{eqnarray}
Therefore, we can measure the optical phase difference as the amplitude changes using a Photo Detector (PD) and detect GWs.

\subsection{Static Response}
Consider the error of the interferometric strain measurement. Bacause the optical phase $\phi_{-}$ is given by
\begin{eqnarray}
  \phi_{-}=\frac{4\pi{L_{-}}}{\lambda},
\end{eqnarray}
where $L_{-}$ is the differential length changes of its arms and $\lambda$ is the wavelength of the input laser, the strain $h$ is represented as 
\begin{eqnarray}
  h = \frac{\Delta{L_{-}}}{L} = \frac{\lambda}{4\pi{L}}\Delta{\phi_{-}} + \frac{L_{-}}{L}\left(\frac{\Delta{f}}{f}\right). \label{eq:eq133_a}
\end{eqnarray}
Moreover, according to Eq.(\ref{eq:eq133}), because infinitesimal change of the optical phase $\Delta{\phi_{-}}$ is given by 
\begin{eqnarray}
  \Delta{\phi_{-}} = \frac{\tan{(\phi_{-})}}{2} \left[\left(\frac{\Delta P_{\mathrm{AS}}}{P_{\mathrm{AS}}}\right) + \left(\frac{\Delta{P_0}}{P_0}\right) \right],
\end{eqnarray}
where $\Delta{P_0}$ is the fluctuation of the input laser and $\Delta{P_{\mathrm{AS}}}$ is a power fluctuation at AS port, finaly, we get a strain as a function of several fluctuation of parameters below;
\begin{eqnarray}
  h = \frac{\lambda}{8\pi{L}}\tan{(\phi_{-})} \left[\left(\frac{\Delta P_{\mathrm{AS}}}{P_{\mathrm{AS}}}\right) + \left(\frac{\Delta{P_0}}{P_0}\right) \right] + \frac{L_{-}}{L}\left(\frac{\Delta{f}}{f}\right). \label{eq:eq137b}
\end{eqnarray}

According to Eq.(\ref{eq:eq137b}), in order to increase the interferometric strain measurement, we should do below;
\begin{itemize}
  \setlength{\itemsep}{1pt}      %2. ブロック間の余白
  \setlength{\parskip}{-1pt}     %4. 段落間余白.
  \setlength{\itemindent}{0pt}   %5. 最初のインデント
  \setlength{\labelsep}{5pt}     %6. item と文字の間
\item we should expand the baseline length $L$.
\item we should operate the Michelson intereferometer at dark fringe, which means $\phi_{-}\to0$, in order to decrease the noise contribution from $(\Delta P_{\mathrm{AS}}/P_{\mathrm{AS}})$ and $\Delta{P_0}/P_0$ to the strain $h$.
\item we should use asymmetric arm so that $L_{0}\to0$, in order to decrease the noise contribution from the laser frequency fluctuation $\Delta{f}{f}$
\end{itemize}

\newpage
\section{Enhancement of the sensitivity}
In order increase the sensitivity, current interferometric GW detector use the Dual-Recycled Fabry-Perot Michelson Interferometer (DRFPMI). 

\begin{figure}[t]
  \begin{center}   
    \includegraphics[width=14cm]{./img_chap1/img133.png}
    \caption{Configuration of interferometric GW detector. (a) Michelson interferometer (MI) (b) Michelson interferometer with two Fabry-Perot optical cavities (FPMI). (c) Dual-Recycled FPMI (DRFPMI)} \label{img:img133}
  \end{center}
\end{figure}


\subsection{Fabry-Perot Michelson Interferometer (FPMI)}

\subsubsection{Fabry-Perot Optical Cavity}
Consider the reflectivity of the Fabry-Perot optical cavity composed of two mirrors separated by L as shown in Fig.\ref{img:img133a}. In this figure, $E_{\mathrm{in}},\,E_{\mathrm{r}},\,E_{\mathrm{t}},\,E$ are the incident, reflected, and transmitted fields respectively, $r_{j}$ and $t_{j}$ are the amplitude reflectivity and transsivity of $j$-th mirrors ($j=1,2$). The reflectivity of the optical cavity $r_\mathrm{{FP}}$ is given by
\begin{eqnarray}
  L_{\mathrm{FWHM}} = \frac{\lambda}{2\mathcal{F}} \\
  \mathcal{F}=\frac{\pi \sqrt{r_{1} r_{2}}}{1-r_{1} r_{2}}
\end{eqnarray}
where $k$ is the wave number \cite{izumi2012multi}.



\begin{figure}[h]
  \begin{minipage}[b]{0.5\hsize}
    \begin{center}   
      \includegraphics[width=7cm]{./img_chap1/img133a.png}
      \subcaption{Fabry-Perot optical cavity composed of two mirrors separated by L. } \label{img:img133a}
    \end{center}
  \end{minipage}\hspace{3pt}
  \begin{minipage}[b]{0.5\hsize}
    \begin{center}   
      \includegraphics[width=5cm]{./img_chap1/img133b.png}
      \subcaption{Intra-cavity power as a function of displacement of cavity length.} \label{img:img133b}
    \end{center}    
  \end{minipage}
  \caption{Fabry-Perot optical cavity.}
\end{figure}


\subsection{Dual-Recycled FPMI (DRFPMI)}
\subsubsection{Power Recycle}
基線長を伸ばすほかに、感度の向上と観測帯域をチューニングするためにFig.\ref{img:img}のような場所にパワーリサイクリング鏡とシグナルリサイクリング鏡と呼ばれる鏡をおいて共振器をつくる\cite{meers1988recycling}。1つ目のパワーリサイクリング鏡は干渉計からレーザーに戻る光を折り返してレーザーパワーを上げるための鏡である。レーザーパワーが上がると、後述するショットノイズとひずみ信号のSNRが向上する。
\subsubsection{Signal Recycle}
次に2つ目のシグナルリサイクリング鏡はASポートに漏れ出たひずみ信号を増幅させるための鏡である。

\subsection{Noise}
\subsubsection{Shot Noise}
テストマスが自由質点として外乱を受けていない理想的な場合には、Michelson干渉計のレーザー光の強度ゆらぎと周波数ゆらぎ、検出器でのショットノイズが感度を制限する。

光検出器で光のパワーを検出する場合、ショットノイズと呼ばれる、光子数のゆらぎに起因するノイズをもつ。光子数のカウントはポアソン分布にしたがうが、光子数$N$が十分に大きい場合標準偏差$\sqrt{N}$のガウス分布に従う。つまり光検出器にパワー$P$の光が入射した場合、このショットノイズは
\begin{eqnarray}
  P_{\mathrm{shot}} \propto \sqrt{P}\ \ [W/\sqrt{\mathrm{Hz}}]  \label{eq:eq136}
\end{eqnarray}
のようなホワイトノイズをもち、ASポートでのパワーの平方根に比例したノイズをもつ。

相対誤差はパワーの平方根に逆比例する。さらにEq.(\ref{eq:eq133})より、光検出の相対誤差は
\begin{eqnarray}
  \frac{\Delta P_{\mathrm{AS}}}{P_{\mathrm{0}}}  \propto \frac{1}{\sqrt{P_{\mathrm{AS}}}}\ \ [1/\sqrt{\mathrm{Hz}}]  \label{eq:eq136}
\end{eqnarray}
にように、Michelson干渉計へ入射した光量の平方根に反比例することがわかる。つまりショットノイズを抑えるためには入射レーザーパワーを上げることが求められる。

\subsubsection{Seismic Noise}
地面振動は地上の重力波検出器においてもっとも振幅が大きい外乱である。さまざまな励起源からの弾性波が地面や構造物をつたわってテストマスを揺らす。そのため地面振動を低減するには、励起源から離れた静かな場所でテストマスを防振することが必要である。

\subsubsection{Newtonian Noise}
Newtonian Noise は、重力勾配ノイズとも呼ばれ、周囲の物体の密度ゆらぎが重力相互作用でテストマスを揺らすノイズである。この密度ゆらぎは地面を伝わる弾性波によって生じ、それが遠隔作用で空間を伝わるため防振することはできない。現在の第二世代の重力波検出器の感度では問題にはならないが、第3世代では10Hz周辺の感度を制限するとされている。

Newtonianノイズを低減するには、地震計アレイをもちいたFeedforward制御が提案されている。

\subsubsection{Thermal Noise}
外部からの外乱以外にも、鏡の基材や表面の粒子がランダムな熱運動をして変位雑音を生み出す。この熱雑音は、1) mirror thermal noise 2) mirror coating thermal noise 3) suspension thermal noise の3つにわけることができる\cite{dan2016study}。

温度$T$をもつ鏡の Mirror thermal noise の変位雑音は、
\begin{eqnarray}
  G_{\mathrm{SB}}(f)=\frac{4 k_{B} T}{\omega} \frac{1-\sigma^{2}}{\sqrt{\pi} E w_{0}} \phi_{\mathrm{sub}}(f)
  \label{eq:eq140}
\end{eqnarray}
のように与えられる\cite{levin1998internal,numata2003wide}。ここで$k_{B}$はボルツマン定数、$\omega$は角周波数、$\sigma$,$E$は基材のポアソン比、ヤング率であり、$\phi_{\mathrm{sub}}$ は鏡の基材のmechanical loss angle、$\omega_0$はビーム半径である。Eq.(\ref{eq:eq140})からわかるように、この熱雑音を低減するには温度を下げるかビーム径を大きくすればよい。

鏡の熱雑音は基材よりもむしろ表面のコーティングで制限される。Corting thermal noise の変位雑音は、
\begin{eqnarray}
G_{\mathrm{CB}}(f)=G_{\mathrm{SB}}(f)\left(1+\frac{2}{\sqrt{\pi}} \frac{1-2 \sigma}{1-\sigma} \frac{\phi_{\mathrm{coat}}}{\phi_{\mathrm{sub}}} \frac{d}{w_{0}}\right)
\end{eqnarray}
で与えられる\cite{numata2003wide,harry2002thermal}。ここで、$d$,$\phi_{\mathrm{coat}}$はそれぞれコーティングの厚さ、loss angle である。

そして最後に、
\textcolor{red}{(サスペンションサーマルノイズを書く。揺動散逸定理を引用して。)}

\section{Large-scale Terrestrial Laser Interferometers}
\subsection{Terrestrial Laser Interferometers}
ひずみに対する感度をあげるには長期線化は必須であり、地上のレーザー干渉計は最終的には数10kmスケールのものが建設される予定である\cite{}。この検出器は第3世代と呼ばれ、

現在までにさまざまな検出器がつくられてきた。それらをTable\ref{tb:tb101}に示す\cite{chen2017brief}。
\begin{table}[h] 
  \begin{center}
    \caption{\cite{chen2017brief,beker2013low}}\label{tb:tb101}
    \begin{tabular}{llll} 
      \hline
      Project & Baseline [km] & Effective Length [km] & Bedrock \\ \hline \hline
      LISM  & 0.02    & 32  & Granite/gneiss \\
      CLIO  & 0.1   & 190 & Granite/gneiss \\
      TAMA  & 0.3   & 96  & - \\ 
      GEO   & 0.6   & 1.2 & Sedimentary rock \\
      KAGRA   & 3  & 2850  & Granite/gneiss \\
      LIGO L1 & 4  & 1150  & Sedimentary soil \\
      LIGO H1 & 4  & 1150  & Sedimentary rock \\
      Virgo   & 3  & 850   & Sedimentary rock \\
      ET      & 10 & 3200  & - \\
      \hline
    \end{tabular}
  \end{center}
\end{table}


\subsubsection{1st Generation}
第1世代検出器(TAMA\cite{ando2001stable}, GEO\cite{grote2010geo}, LIGO\cite{abbott2009ligo}, Virgo\cite{accadia2012virgo})は、これまでの実験室スケールの干渉計とは異なり、はじめての大型検出器であった。これらの干渉計はパワーリサイクルをしたFPMIであり、

\subsubsection{2nd Generation}
第2世代検出器(KAGRA\cite{akutsu2018kagra}, Advanced Virgo\cite{acernese2014advanced}, Advanced LIGO\cite{aasi2015advanced})では、

\subsubsection{3rd Generation}
第3世代検出器では基線長をさらに10倍長くする。

%% \subsubsection{LISM}
%% LISMは世界で初めて地下に建設された20mのレーザー干渉計型の重力波検出器をつかって、干渉計の安定した運転を地下環境で実証するプロジェクトである。LISMの干渉計は20mのFabry-Perot光共振器を腕にもつマイケルソン型レーザー干渉計である。この腕共振器のフィネスは25000と非常に大きくCavity Pole は150Hzに相当する。このようにフィネスの大きい共振器にもかかわらず、DutyCycleは$99.8\,\%$であった。

%% 安定稼働の大きな要因としては地面振動の同相雑音除去効果が考えられる。Fig.\ref{img:img122}にLISMの感度曲線を示す。懸架システムの伝達関数をもちいて計算された干渉計への地面振動の寄与と比較すると、地面振動の垂直方向成分が100Hz以下の感度を制限していることがわかるが、特筆すべきは6Hz以下の干渉計の感度が地面振動の寄与よりも小さいことである。これは、神岡の地下岩盤が十分に固く、低周波では20mの基線を一枚岩のようにうごかし基線長が低減される。このおかげで腕共振器が安定してロックでき、安定した干渉計の稼働が実現できた。
%% \begin{figure}[h]
%%   \begin{center}   
%%     \includegraphics[width=12cm,height=7cm]{./img_chap1/img122.png}
%%     \caption{The noise equivalent detector sensitivity of LISM. This figure is cited from figure 5 in \cite{sato2004ultrastable}. } \label{img:img122}
%%   \end{center}
%% \end{figure}

%% \subsubsection{CLIO}
%% CLIOは、LISMに引き続いて地下に建設された100mのレーザー干渉計型重力波検出器であり、この検出器の目的は極低温に冷やした鏡を用いて熱雑音の低減を実証することである\cite{ohashi2003design}。そのために、地面振動が静かな地下環境で、熱伝導特性の良いサファイアでFabry−Pert光共振器を構築し、低振動なパルス管冷凍機\cite{tomaru2004development}を使用して20Kまで冷却をしている。このおかげで、室温での熱雑音で制限されていた干渉計の感度を低音鏡にして低減することができた\cite{uchiyama2012reduction}。

%% \begin{figure}[h]
%%   \begin{center}   
%%     \includegraphics[width=12cm,height=7cm]{./img_chap1/img123.png}
%%     \caption{This figure is cited from figure 2 in \cite{uchiyama2012reduction}. } \label{img:img123}
%%   \end{center}
%% \end{figure}

\subsection{KAGRA}
KAGRAは第2世代の干渉計であるが、他の検出器とは違い、第3世代に必要な地下環境と極低温鏡の技術を採用しており、第2.5世代とも呼ばれている。

\subsubsection{KAGRA}



\section{Summary of the Chapter}
 % Background
%\chapter{Geophysics Interferometer (GIF)}



\section{Overview} 
Geophysic Interferometer とはなにか述べる。



\section{Purpose} % GIFの目的
この干渉計の目的を述べる。どういう地球物理の現象をターゲットにしているかとか述べる。それをもとに要求値が決まっていく。
\subsection{Motivation in Geophysics}
地物でのモチベーション。(ここは新谷さん高森さん早河さんに聞いて書く。)
\subsection{Motivation in GW detectors}
重力波望遠鏡でのモチベーション。Drever氏のSPIの原論文で述べられていた「Seismic Interferometer」のように基線長伸縮を低減することが、GIFをKAGRAに設置するモチベーション。




\section{Working Principle} %動作原理 
動作原理を書く。どうやって地面の歪みをマイケルソン干渉計で検出するのかここに書く。
\subsection{Response to the seismic strain}
どうやって地面の歪みが基線長伸縮として応答するか述べる。地面の歪みから基線長伸縮への伝達関数を載せる。
\subsection{Signal Detection Scheme}
どうやって干渉計信号から基線長伸縮を取り出すか述べる。Quadrature Phase Detection について書く。ここからサンプリング周波数への要求値が定まることを述べる。
\subsection{Noise}
どういうノイズが原理的に存在するか述べる。空気ゆらぎ、周波数雑音を述べる。




\section{Optics} %光学系
どうやって実際の干渉計を構築しているか述べる。
\subsection{Mode Matching Optics}
どういうモードマッチをして干渉計として光を干渉させているか述べる。
\subsection{Frequency Stabilized Laser}
どういう制御をして周波数安定をしているか述べる。
\subsection{Core Optics}
\subsubsection{Beam Splitter}
どういうミラーを使っているか述べる。
\subsubsection{Corner Cube}
どういうミラーを使っているか述べる。大きさとか表面の精度とか。




\section{Data Aquisition System} %DAQ
DAQについて述べる。冗長性を持たせるために二系統のDAQを使っていることを述べる。一方はKAGRAとは独立で、もう一方はKAGRAと同じシステムに組み込んでいることを述べる。
\subsection{Stand Alone System}
森井システムについてのべる。コンパクトなシステムで地下環境でも安定して動くシステムだ、と述べる。
\subsection{Realtime System}
KAGRAのリアルタイムシステムについて述べる。KAGRAの制御に組み込むために歪変換をリアルタイムで行っている、と述べる。




\section{Summary of the Chapter} %章のまとめ
本章で述べたパラメータを表にまとめる。
 % KAGRA Interferometer
\chapter{Geophysics Interferometer (GIF)}



\section{Overview} 


\section{Purpose}
\subsection{Motivation in Geophysics}
\subsection{Motivation in GW detectors}


\section{Working Principle}

\subsection{Asynmetric Michelson Interferometer}
\begin{eqnarray}
  \phi = 2\pi\frac{2(l_x-l_y)}{\lambda}\sim4\pi\frac{l_x}{\lambda}
\end{eqnarray}

\begin{eqnarray}
  |d\phi| = 4\pi\frac{l_x}{\lambda}\left( \left|\frac{d\lambda}{\lambda}\right| + \left|\frac{dl_x}{l_x}\right| \right)
\end{eqnarray}


\subsection{Response to the seismic strain}
\begin{figure}[h]
  \begin{center}
    \includegraphics[width=10.0cm]{./img_chap2/img210.png}
    \caption{The displacements of the two points which are sparated L in X axis. }
  \end{center}
\end{figure}

The response of the strainmeter to seismic waves have characteristics of the low pass filter. To calculate this response, it is assumed that the plane seismic waves which displacement $u(x,t)$ is represented as $u(x,t)=u_0e^{i(\omega{t}-kx)}$ with angular frequency of $\omega$ and wave number of $k$, propagate along with the direction of the base-line of the strainmeter. The length fluctuation between two mirrors sparated with $L$ can be expressed as 
\begin{eqnarray}
  \Delta{L(t)} &\equiv& u(0,t) - u(L,t) \\
  &=& u(0,t) - u(0,t-\tau), \label{eq:chap2_10}
\end{eqnarray}
where $\tau=L/v$ is the time delay. 
The transfer function from the displacement to the length fluctuation is
\begin{eqnarray}
  H_{\mathrm{disp}}(s) \equiv \frac{\Delta{L(s)}}{u(s)} = 1 - \mathrm{exp}(-\tau{s})
\end{eqnarray}

Because the strain amplitude $\epsilon(x,t)$ is defined as $\epsilon(x,t)\equiv\frac{du}{dx}$, the strain
\begin{eqnarray}
  \epsilon(x,t) \equiv \frac{du}{dx} &=& \frac{du}{dt} \frac{dt}{dx}\\
  &=& {u(x,t)}^{\prime}\frac{1}{v}
\end{eqnarray}


Therefore, the response of the strainmter to the seismic strain is given

\begin{eqnarray}
  H_{\mathrm{strain}}(s) \equiv \frac{\Delta{L(s)}}{\epsilon(s)} = \frac{\Delta{L(s)}}{\frac{s}{v}u(s)} = \left(1 - \mathrm{exp}(-\tau{s})\right) \frac{v}{s}
\end{eqnarray}



\begin{figure}[h]
  \begin{center}
    \includegraphics[width=10.0cm]{./img_chap2/img211.png}
    \caption{}
  \end{center}
\end{figure}


\begin{figure}[h]
  \begin{center}
    \includegraphics[width=12.0cm]{./img_chap2/img212.png}
    \caption{}
  \end{center}
\end{figure}


\subsection{Signal Detection Scheme}
\subsubsection{Quadrature Phase Detection}


\subsection{Noise}
どういうノイズが原理的に存在するか述べる。空気ゆらぎ、周波数雑音を述べる。




\section{Optics} %光学系
どうやって実際の干渉計を構築しているか述べる。
\subsection{Mode Matching Optics}
どういうモードマッチをして干渉計として光を干渉させているか述べる。
\subsection{Frequency Stabilized Laser}
どういう制御をして周波数安定をしているか述べる。
\subsection{Core Optics}
\subsubsection{Beam Splitter}
どういうミラーを使っているか述べる。
\subsubsection{Corner Cube}
どういうミラーを使っているか述べる。大きさとか表面の精度とか。




\section{Data Aquisition System} %DAQ
DAQについて述べる。冗長性を持たせるために二系統のDAQを使っていることを述べる。一方はKAGRAとは独立で、もう一方はKAGRAと同じシステムに組み込んでいることを述べる。
\subsection{Stand Alone System}
森井システムについてのべる。コンパクトなシステムで地下環境でも安定して動くシステムだ、と述べる。
\subsection{Realtime System}
KAGRAのリアルタイムシステムについて述べる。KAGRAの制御に組み込むために歪変換をリアルタイムで行っている、と述べる。




\section{Summary of the Chapter} %章のまとめ
本章で述べたパラメータを表にまとめる。
 % Underground Seismic Noise
\chapter{Geophysics Interferometer (GIF)}
KAGRA is the only GW detector, which has a strainmeter to monitor its baseline length changes. The strainmeter is named Geophysics interferometer (GIF).

GIF is a laser interferometric strainmeter, which is developed by Earthquake Research Institute, University of Tokyo. The purpose of the strainmeter is to observe not only the earthwuake but also the earth free oscillation. Unlike a seismometer, the strainmeter has sensitivity in low-frequency. Moreover, unlike the continuous GPS (CGPS) nets, which also measures a strain ($\sim\,10^{-8}$), the strainmeter has more precision ($\sim\,10^{-12}$) \cite{araya2007broadband}.

In this chapter, instruments of GIF are described. After overview of GIF in section \cref{sec:sec41}, working principles of the interferometer are described in section \cref{sec:sec42}. Optics of GIF are described in section \cref{sec:sec43}. Realtime signal aquisition system to send the strain signal to KAGRA is described in \cref{sec:sec44}

\section{Overview} \label{sec:sec41}
Geophysics interferometer (GIF) is a $1500\,\mathrm{m}$ laser strainmeter constructed along the X-arm baseline of KAGRA. As shown in Fig.\ref{img:img402}, GIF is an asymmetric Michelson interferometer unlike KAGRA interferometer. Moreover, mirrors of the interferometer of GIF are fixed on the ground to monitor the baseline length changes directly. GIF is now only installed on the X-arm. GIF have been observing the baseline changes for almost 3 years.
\begin{figure}[h]
  \centering
  \includegraphics[width=8cm]{./img_chap4/img402.png}
  \caption{Location of geophysics interferometer (GIF). Whereas KAGRA is a symmetric L-shape $3000\,\mathrm{m}$ Michelson interferometer, GIF is an asymmetric $1500\,\mathrm{m}$ Michelson interferometer. GIF is only installed along the X-arm tunnel.} \label{img:img402}
\end{figure}

\section{Working Principle} \label{sec:sec42}
As described in section \cref{sec:12}, working principle of the strain measurement of GIF is the same as the GW detectors. However, the sensitivity of GIF is limited by the laser frequency noise due to the {\it asymmetric} two arm length.
\subsection{Asymmetric Michelson Interferometer}
\begin{figure}[h]
  \centering
  \includegraphics[width=10.0cm]{./img_chap4/img401.png}
  \caption{Schematic drawing of the GIF as an asymmetric Michelson interferometer, which has two different arm length, $l_x\gg{l_y}$. In this figure, the mode matching optics and the optics for signal detection are not drawn.} \label{img:img401}
\end{figure}

Fig.\ref{img:img401}に非対称マイケルソン干渉計であるGIFの略図を示す。GIFは$70\,\mathrm{cm}$の腕の長さ変化を基準として$1500\,\mathrm{m}$の腕の変化をREFL位置で干渉信号を取得する。

ここで、腕の非対称さがひずみ計測にどのように影響をあたえるか考える。長さ変化による干渉光の位相$\phi$とマイケルソン干渉計の腕の差動成分${L_{-}}=L_{\mathrm{x}}-L_{\mathrm{y}}$には、レーザーの波長を$\lambda$とすれば、${\phi}_{-} = 4\pi\frac{{L_{-}}}{\lambda}$の関係がある。そしてそれらの微小変化には
\begin{eqnarray}  
  \left|\Delta \phi_{-}\right| = \frac{4\pi{L_{-}}}{\lambda}\left( \left|\frac{\Delta L_{-}}{L_{-}}\right| + \left|\frac{\Delta f}{f}\right|\right) \label{eq:eq400}
\end{eqnarray}
の関係がある。なお、$f$はレーザーの周波数であり、 $|\frac{\Delta{\lambda}}{\lambda}| = |\frac{\Delta{f}}{f}|$の関係をつかった。

ここで、2つの腕の長さ$L_{\mathrm{x}},\,L_{\mathrm{y}}$が十分に非対称、つまり$L_{\mathrm{x}} \gg L_{\mathrm{y}}$とし、また短い方の腕の長さ$L_{\mathrm{y}}$の変動が無視できるとする, because the reference arm of the interferometer is made by a super invar board. Therefore, 腕の差動成分は$\L_{-}\sim{L_{\mathrm{x}}}$と表すことができる。Eq.(\ref{eq:eq400})は、X腕のひずみ$h=\Delta{L_{\mathrm{x}}}/L_{\mathrm{x}}$をつかって
\begin{eqnarray}  
  \left|\Delta \phi_{-}\right| = \frac{4\pi{L_{\mathrm{x}}}}{\lambda}\left( \left|h\right|  + \left|\frac{\Delta f}{f}\right|\right) \label{eq:eq400_a}
\end{eqnarray}
となる。This equation shows that 周波数ゆらぎ$|\frac{\Delta{f}}{f}|$はそのままひずみ計測のノイズとなる。

\subsection{Seismic Strain Response}
\subsubsection{Response from $u$ to $\Delta{L}$ ($H_{\mathrm{disp}}$)}
In order to calculate the response from the seismic strain to the optical phase of the GIF interferometer, the same as the Fig.(\ref{img:img310}), it is assumed that the plane seismic waves which displacement $u(t,x)$ is represented as $u(t,x)=u_0e^{i(\omega{t}-kx)}$ with angular frequency of $\omega$ and wave number of $k$, which propagates along with the direction of the baseline of the strainmeter (right direction in this figure). First, because the length fluctuation between two mirrors sparated with $L$ can be expressed as 
\begin{eqnarray} 
  \Delta{L(t)} &\equiv& u(t,0) - u(t,L) \\
  &=& u(t,0) - u(t-\tau,0), \label{eq:eq403}
\end{eqnarray}
where $\tau=L/v$ is the time delay, the transfer function from the displacement to the length fluctuation is given by Laplace transform as
\begin{eqnarray} \label{eq:eq404}
  H_{\mathrm{disp}}(s) \equiv \frac{\Delta{L(s)}}{u(s)} = \frac{u(s)\left[ 1-\exp(-\tau{s}) \right]}{u(s)} = 1 - \exp(-\tau{s})
\end{eqnarray}
\subsubsection{Response from $\epsilon$ to $\Delta{L}$ ($H_{\mathrm{strain}}$)}
Moreover, because the strain amplitude $\epsilon{(x,t)}$ is defined as $\epsilon{(x,t)}\equiv\frac{du}{dx}$, the seismic strain is represented as 
\begin{eqnarray} 
  \epsilon{(x,t)} \equiv \frac{du}{dx} = \frac{du}{dt} \frac{dt}{dx} =  \frac{du}{dt} \frac{1}{v} \label{eq:eq406}
\end{eqnarray}
Therefore, similarly, the transfer function from the seismic strain to the displacement is given  as
\begin{eqnarray} \label{eq:eq407}
  u(s) = \frac{v}{s} \epsilon(s).
\end{eqnarray}
Finary, because the transfer function from the length change of the baseline to the optical phase is given as $4\pi/{\lambda_{\mathrm{opt}}}$, the transfer function from the seismic strain to the optical phase is represented as 
\begin{eqnarray} \label{eq:eq407}
  H_{\mathrm{strain}}(s) = 4\pi\frac{1}{\lambda_{\mathrm{opt}}} \left[1 - \exp(-\tau{s}) \right]\frac{v}{s}.
\end{eqnarray}
Here, as a summary of these transfer function, these are related with each other as shown in Fig.(\ref{img:img411}). 

\subsubsection{Improvement of the sensitivity with longer baseline}
\begin{figure}[h]
  \centering
  \includegraphics[width=10.0cm]{./img_chap4/img411.png}
  \caption{The response from seismic strain to optical phase. $\epsilon$ is the seismic strain, $u$ is the displacement, $\Delta{L}$ is the length change of the X-arm baseline, and $\phi$ is the optical phase of the GIF interferometer. $C$ is the optical gain of the GIF interferometer given in Eq.(\ref{eq:eq401}). $H_{\mathrm{disp}}$ is the transfer function from the displacement to the length change given in Eq.(\ref{eq:eq404}). $v/s$ is the transfer function from the seismic strain to the displacement given in Eq.(\ref{eq:eq406}). } \label{img:img411}
\end{figure}

Eq.(\ref{eq:eq407})で表される、基線長の異なる2つのMichelson干渉計のひずみから位相への伝達関数のボード図をFig.\ref{img:img411_a}に示す。長さが倍になるとDCでのゲインも2倍になる。コーナー周波数$f_0\equiv \frac{1}{\tau}$は
\begin{eqnarray}
  f_0 = \frac{v}{L}
\end{eqnarray}
で表せるので、長さが二倍になるとコーナー周波数は半分になり、帯域が減ることもわかる。基線長が$1500\,\mathrm{m}$のGIFの場合、弾性波速度を$5.5\,\mathrm{km}$とすれば、コーナー周波数は$f_0\sim3.7\,\mathrm{Hz}$である。つまりそれ以下の帯域ではGIFは歪に対して平坦な応答を示す。

\begin{figure}[p]
  \begin{center}
    \includegraphics[width=13.0cm]{./img_chap4/img412.png}
    \caption{Compasison of the transfer function from strain of the baseline $\epsilon$ to the length change of that $\Delta{L}$ in the different baseline length. $3000\,\mathrm{m}$ の基線長ではその半分の$1500\,\mathrm{Hz}$よりも、DCゲインは二倍大きい一方で、コーナー周波数は \color{red}{A} になり帯域が減る。また、周波数が \color{red}{B} の条件を満たすとき、ゲインはゼロになる。なぜならば、このときひずみは基線を同相で動かし、基線長伸縮として現れないためである。}\label{img:img411_a}
  \end{center}
\end{figure}

\subsection{Noise}
\subsubsection{Frequency Noise}
先述したように、GIFのような1500mと70cmの腕を持つ非対称マイケルソン干渉計は、腕の同相雑音除去が効かない。周波数ノイズは
\begin{eqnarray}
  h = \frac{\Delta{f}}{f} \sim 7\times10^{-13} [\mathrm{1/\mathrm{Hz}}]
\end{eqnarray}
になる\cite{araya2017design}。

\subsubsection{Residual Gas Noise}
Bcause residual gas fluctuates the optical path, length measured by interferometer is also fluctuates. The opttical path $L_{\mathrm{opt}}$ is given by $L_{\mathrm{opt}}=nL$, where $L$ is the length of the baseline and $n$ is the refraction index in the optical path relative to the path in the vacuum. Under the pressure of $p$ in vacuum, the index $n$ is approximated as $n = 1 + c_0(p/p_0)$, where $c_0$ denotes the relattive refractive index, $p_0$ is pressure in standard air at 1 atm. The apparent strain due to the residual pressure is given as \cite{ciddor1996refractive};
\begin{eqnarray}
  h = (L_{\mathrm{opt}}-L)/L = c_0(p/p_0) \sim 3\times10^{-9} p.
\end{eqnarray}
In order to maintain the strain sensitivity; $3\times10^{-13}$, the vacuum pressure should be below $1\times10^{-4}\,[\mathrm{Pa}]$. However, actual vacuum pressure is $1\times10^{-2}\,[\mathrm{Pa}]$, then strain is $\sim\times10^{-12}$.



\section{Optics} \label{sec:sec43} %光学系
これまでの議論はレーザー光を平面波として扱っていた。しかしながら実際のレーザー光は伝搬する距離で位相のずれやビームサイズの変化が生じる。このようなビームを有限の範囲でかんしょうさせるにはこれらビームプロファイルを適切に設計しなければならない。ここでは、レーザー光がガウシアンビームだとして、$1500\mathrm{m}$の非対称マイケルソン干渉計を干渉させるために必要なモードマッチについて述べる。

\subsection{Gaussian Beam}
理想的なレーザー光は$\mathrm{TEM}_{00}$と呼ばれる空間モードをもち、電場の位相は距離に応じて変化する。この空間モードをもつビームのことをガウシアンビームと呼ぶ。このガウシアンビームが$z$軸に伝搬する場合を考える。この電場は
\begin{eqnarray}
  u(x, y, z)=\sqrt{\frac{2}{\pi{w^2(z)}}} \exp \left(i\zeta(z)-\mathrm{i} k \frac{x^{2} +y^{2}}{2 R(z)}-i\frac{2\pi}{\lambda}z\right)
  \exp \left(-\frac{x^{2}+y^{2}}{w^{2}(z)}\right)  \label{eq:eq415}
\end{eqnarray}
とかける\cite{bond2016interferometer,svelto1998principles}。ここで、$\lambda,\,w_0$はそれぞれレーザーの波長、$z=0$でのビーム径である。また
\begin{eqnarray}
  z_0 &=& \frac{\pi{w^2_0}}{\lambda} \\ \label{eq:eq415_a}
  w(z) &=& w_0\sqrt{1+\left(\frac{z}{z_0}\right)^2}, \\ \label{eq:eq415_b}
  R(z) &=& z\left[1+\left(\frac{z_0}{z}\right)^2\right],\\ \label{eq:eq415_c}
  \phi(z) &=& \arctan\left(\frac{z}{z_0}\right) \label{eq:eq415_d}
\end{eqnarray}
はそれぞれ、Reyliegh length、$z$でのビーム径、曲率、Gouy位相である。このときEq.(\ref{eq:415})から、Fig.\ref{img:img415a}にしめすように、ガウシアンビームのパワー$P=|u^2|$はガウス分布をもつことがわかる。さらにビーム径はビーム強度が$1/e^2$になるときの半径とわかる。
\begin{figure}[p]
  \begin{minipage}{14cm}
    \centering    
    \includegraphics[width=8cm]{./img_chap4/img415a.png}
    \subcaption{Evolution of a Gaussian beam propagating along the z-axis\cite{riehle2006frequency}}{$w_0$ denotes a beam radius at beam weist, where $z=0$. $w(z)$ and $R(z)$ are the beam radius and curvature at $z$. Gouy phase is not shown in here.}\label{img:img415a}
  \end{minipage}\\
  \begin{minipage}{14cm}
    \centering        
    \includegraphics[width=14cm]{./img_chap4/img415.png}
    \subcaption{Beam prifile}{(left) Beam radius normalized by $w_0$ as a function of $z/z_0$, where $z_0$ is Rayleigh length. (Middle) Beam curvature normalized by $z_0$. (right) Gouy phase.}\label{img:img415}    
  \end{minipage}
  \caption{Gaussian beam.}
\end{figure}


ガウシアンビームを特徴づける$z$の関数であるパラメータEq.(\ref{eq:eq415_b,eq:eq415_c,eq:eq415_d})をFig.{\ref{img:img415}}に示す。$z=0$のとき、ビーム径は最も小さくEq.(\ref{eq:eq415})の位相は0であるため、ガウシアンビームは平面波とみなせる。一方で、$z\gg{z_0}$のときレーザー光源は点光源とみなせ、球面波としてふるまう。


\subsection{Reflector Design}
リフレクタの大きさを最小限にするために、GIFの干渉計はエンドミラーでビーム直径が最も小さくなるビームウエストがくるようにしている。この場合、ビームウエスト$w_0$を小さくしたいが、小さくしすぎると$L=1500\,\mathrm{m}$離れたBSとフロントリフレクタで大きくなるので、できるだけフロントでのビーム径$w(L)$はエンドのビーム径$w_0$に対してできる限り小さくしたい。つまりこれを式で表すと、
\begin{eqnarray}
  \argmin_{w_0} \left[w_0\times\frac{w(L)}{w_0}\right] \label{eq:eq415_e}
\end{eqnarray}
となるような$w_0$を探せばよい。Eq.(\ref{eq:eq415_b})をEq.(\ref{eq:eq415_e})に代入して解けば
\begin{eqnarray}
  w_0 = \sqrt{\frac{{L\lambda}}{\pi}}
\end{eqnarray}
を得る。つまりビームウエストサイズ$w_0=\sqrt{{1500\,\mathrm{[m]}}\times 532\,\,\mathrm{[nm]}/\pi} = 16\,\mathrm{mm}$となる。このときのフロントリフレクタでのビーム径は$w(L)=\sqrt{2}{w_0}$になる。ちなみに、リフレクタの大きさはフロントリフレクタでのビーム径の3倍の大きさを往復できるようにするには、最低限必要なリフレクタの aperture diameter は$2\times3\times\sqrt{2}w_0\sim270\,\mathrm{mm}$となる。

\subsection{Input Output Optics}
レーザー光源から出射されたビームを適当な大きさにして干渉計へ入射するために、input output optics と呼ばれる光学系を構築している。Fig(\ref{img:img416})にGIFのinput output optics と干渉計を示す。光源からの出射ビームは、エンドリフレクタの位置Aでビームウエストになるように、コリメータ(1)とステアリングミラー(2)、凹面鏡(3)を経てBSへと入射される。2つのリフレクタから反射してきたビームは地点Bで再結合し、2つ目の凹面鏡とコリメータ(4)をへてPDに入射する。これらopticsの調整をおこない干渉信号を得ている\cite{miyo2017baseline}。

\begin{figure}[h]
  \begin{center}   
    \includegraphics[width=14cm]{./img_chap4/img416.png}
    \caption{Schematic optics layout}{(1) A collimator lens for input beam. (2) A flat mirror for steering mirror. (3) Two concave mirrors with a radius of curvature of $9.8\,\mathrm{m}$ for mode matching. (4) A collimator lens for output beam. The waist of the beam is at the end reflector at point A. Two reflected on the reflectors are combined at point B.}\label{img:img416}
  \end{center}
\end{figure}


\subsection{Core Optics}
The core optics of the Michelson interferometer are composed of two reflectors and beam splitter (BS). 

\begin{figure}[h]
  \begin{minipage}[b]{7cm}
    \begin{center}   
      %\includegraphics[width=7cm]{./img_chap4/img418.png} % ファイル重い
      \includegraphics[width=7cm]{./img.png}
      \subcaption{Core optics in the front vacuum chamber. }\label{img:img418}
    \end{center}
  \end{minipage}\hspace{0.1cm}
  \begin{minipage}[b]{7cm}
    \begin{center}   
      %\includegraphics[width=7cm]{./img_chap4/img419.png} %ファイル重い
      \includegraphics[width=7cm]{./img.png}      
      \subcaption{Core optics in the end vacuum chamber. }\label{img:img419}
    \end{center}
  \end{minipage}
  \caption{}  
\end{figure}


\subsection{Frequency Stabilized Laser}
Because GIF is an asymmetric Michelson interferometer, the frequency stability of the laser would limit the sensitivity of the strain, and we use the frequency stabilized laser, which is stabilized the laser frequency to the iodine absorption line \cite{araya2002iodine}. The control diagram of the frequency stabilization system is shown in Fig.\ref{img:img417}. このシステムはヨウ素分子の吸収スペクトル線の周波数とレーザーの周波数との差を利用したフィードバック制御である。エラー信号は、ポンプ光とプローブ光をつかったドップラーフリーな吸収線信号\cite{snyder1980high}をPDH法をつかって取得する。

\begin{figure}[h]
  \begin{center}   
    \includegraphics[width=12cm]{./img_chap4/img417.png}
    \caption{Schematic diagram of the frequency-stabilization system of the GIF main laser.}\label{img:img417}
  \end{center}
\end{figure}

\section{Realtime Signal  Aquisition System} \label{sec:sec44}



\subsection{Quadrature Phase Fringe Detection}
\begin{figure}[h]
  \begin{center}
    \includegraphics[width=13.0cm]{./img_chap4/img413.png}
    \caption{Quadrature interferometer used in the GIF strainmeter. A half-wave plate (HWP) produces a p-polarization and s-polarization. A quator-wave plate (QWP) delay the optical phase of the s-polarized light with 90 degree against to the another. As a result, one can obtain the quadrature phase fringe.}\label{img:img413}
  \end{center}
\end{figure}

We use the quadrature phase fringe detection to measure the length change of the baseline with wide dynamic range \cite{bobroff1993recent}. The optical layout for the detection is shown in Fig.(\ref{img:img413}).

The quadrature phase fringes are detected by two photo detectors, these can be represented as
\begin{eqnarray}
  x(t) &=& x_0 + a \sin(\phi(t)+\phi_0), \\
  y(t) &=& y_0 + b \cos(\phi(t)),
\end{eqnarray}
where $x$ and $y$ are the two voltage outputs from the detectors, $a$ and $b$ are the amplitudes of these fringe signals, $x_0$ and $y_0$ are the offsets, $\phi$ is optical phase, and $\phi_0$ is the phase offsets from imperfections \cite{zumberge2004resolving}.
このとき、位相角$\phi$は
\begin{eqnarray}
  \tan{\phi(t)} = \frac{1}{{\cos(\phi_0)}} \left(\displaystyle{\frac{b}{a}\frac{x(t)-x_0}{y(t)-y_0}-\sin(\phi_0)}\right)
\end{eqnarray}
で表される。つまりある時刻$t$のときに、パラメーター$x_0,\,y_0,\,a,\,b,\,\phi_0$が与えられれば、そのときの位相$\phi(t)$は求まる。

\subsection{Realtime Data Processing}
KAGRAのデジタルシステムをつかってリアルタイムで楕円パラメータを取得する。KAGRAのデジタルシステムでは

\cite{bork2001overview}


GIFからの2つの干渉信号を

Fig.\ref{img:img420}にひずみ変換のMatlabのSimlinkモデルを示す。

\begin{figure}[h]
  \centering
  \includegraphics[width=15.0cm]{./img_chap4/img420.png}
  \caption{}\label{img:img420}
\end{figure}



\section{Summary of the Chapter} %章のまとめ
本章で述べたパラメータを表にまとめる。
 % Geophysics InterFerometer
\chapter{Arm Length Compensation System for Global Seismic Control}
\section{Introduction}
Seismic noise cause two main problems to the terrestrial gravitational-wave detectors. First one is the limitation of the sensitivity. Amplitude spectrum density of the seismic noise, is empirically kwnon as
\begin{eqnarray}
  \sim \frac{10^{-7}}{f^2}\ \mathrm{m}/\sqrt{\mathrm{Hz}},
\end{eqnarray}
where $f$ is a frequency of the spectrum. This noise limits the sensitivity of the detectors in lower frequency tipicaly below 10 $\mathrm{Hz}$ even after the attenuation by the vibration isolation systems. On the other hands, second problem is the decrease of the duty cycle of the GW detectors. Laser interferometric detector has an Fabry-Perot optical resonant cavities to enhance the sensitivity of GWs. This optical cavity only resonant within the narrow linewidth of few $\mathrm{nm}$, whereas the seismic noise is larger than this width by two orders of magnitudes. 

Underground can resolve these problems. Underground is more quiet than the surface of the ground \cite{carter1991high}. Especially, the underground seismic noise above 1 $\mathrm{Hz}$ is effectivly reduced than the noise on surface of the ground \cite{lcgt2009lcgt}. For example, a laser interferometer gravitational wave antenna with a baseline length of 20 $\mathrm{m}$ (LISM) constructed underground have demonstrated the stable performance of the detector by resulting the high duty cycle of 99.8 $\%$ \cite{sato2004ultrastable}. 

しかし、KAGRAのような3kmの長期線のレーザー干渉計では、LISMのように安定して可動させることは難しいとされている。なぜならば、地面振動による基線長変動は、基線長が長いほどその影響は大きいためである。後述する\cref{sec:33}によれば、0.2Hzの脈動による基線長変動への影響は、KAGRAはLISMの150倍ある一方で、表\ref{tb:301}に示すように、線幅はおよそ17倍しかない。つまりKAGRAはLISMと比べて、線幅に対して地面振動による基線長伸縮はおよそ1桁大きいことを意味する。このような長期線化による問題は、KAGRAなどの第二世代の検出器だけの問題ではなく、ETなどの数10kmの基線長を計画する第3世代の検出器にとって同様の問題となる。

\begin{table}[H]
  \centering
  \caption{Comparison of the line width of the arm cavity}
  \begin{tabular}{lllll}
    \hline
    & Finess   & Line width [$\mu\mathrm{m}$] & Baseline length [m]\\
    \hline
    LISM        & 25000  & 0.021 & 20\\
    KAGRA       & 1500   & 0.35  & 3000\\
    \hline
  \end{tabular}\label{tb:301}
\end{table}


\section{Basics in Vibration Isolation and Control Technique}
\subsection{Passive Vibration Isolation}
\subsubsection{Single Pendulum}
\subsubsection{Multi Pendulum}
\subsection{Active Vibration Isolation}
\subsection{Sensor Belnding Control Technique}
\subsection{2 Types Feedforward Control Techniques}
\subsubsection{Feedforward at Feedback Point} % Sensor Correction
\subsubsection{Feedforward at Error Point} % Feedfoward 

\subsection{Toward the Global Seismic Control}
\subsubsection{Overview}
\subsubsection{Suspension Point Interferometer}

%
\section{Difficulties in the Global Seismic Control}
\subsection{Overview}
\subsection{Actuator Range Limit}
\subsection{...}
\subsection{...}


%
\section{Arm Length Compensation Using Geophysics Interferometer}
\subsection{Concept}
\subsection{Geophysics Interferometer for Sensing the Arm Length}
\subsection{Arm Length Compensation}
\subsection{Requirements}


\section{Summary of the Chapter}
 % Arm Length Compensation for Global Seismic Control
\chapter{Demonstration of Baseline Compensation System} \label{chap5}
In this chapter, the demonstration of the baseline compensation system is described. The purpose of this demonstration is to compensate for the deformation of the baseline so that the length fluctuation of the 3-km arm cavity is reduced below 1 Hz, where the passive seismic isolation cannot attenuate the seismic disturbance. 

In the section \cref{sec:sec51}, experimental arrangement for demonstration is described. In section \cref{sec:sec52}, the result of the test is described. In the end, the discussion is described in section \cref{sec:sec53}.

\section{Experimental Arrangement} \label{sec:sec51}
Because the purpose of the baseline compensation system is to reduce the arm cavity length fluctuation, we prepared the experimental arrangement to measure the length.

\subsection{Measurement of X-arm cavity length}
The length fluctuation of the X-arm cavity is measured by the PDH method \cite{drever1983laser}. This method obtains the error signal, which is proportional to the displacement from the nominal length where the cavity is on resonance. In order to keep the resonance, the error signal is fed back to the acousto-optics modulator (AOM), which changes the input laser frequency.

Brief measurement procedure is shown in Figure \ref{img:img600}. (1) The deformation of the baseline causes the length change of the arm cavity length through the suspensions. Suppose that the baseline length is displaced by $\Delta{L}$ from the nominal length
of $L$. Utilizing the PDH method, we can obtain the error signal proportional to this displacement. (2) This signal is also interpreted as the frequency changes of the input laser because the frequency change $\Delta{f}$ has a relation with the baseline length change $\Delta{L}$ \cite{izumi2012multi};
\begin{eqnarray}
  \displaystyle -\frac{\Delta{f}}{f} = \frac{\Delta{L}}{L}.
\end{eqnarray}
(3) To keep the optical cavity on resonance, the signal is fed back to the AOM, which is the frequency actuator. In this procedure, the length fluctuation is obtained from the feedback signal to the AOM.
\begin{figure}[h]
  \centering
  \includegraphics[width=13cm]{./img_chap6/img600.png}
  \caption{Experimental arrangement for X-arm length measurement. X-arm cavity controled by feeding the PDH signal back to the AOM of the input laser to keep on resonance. The length change of the cavity is obtained from the feedback signal.}
  \label{img:img600}  
\end{figure}



\subsection{Control Design}
\begin{figure}[h]
  \begin{minipage}{14cm}
    \begin{center}   
      \includegraphics[width=14cm]{./img_chap6/img630a.png}
      \subcaption{Schematic contol of each platform stage. Left figure is that of the IX stage, right figure is that of the EX.}\label{img:img630a} \hfill\vspace{10pt}
    \end{center}
  \end{minipage}
  \begin{minipage}{14cm}
    \begin{center}   
      \includegraphics[width=14cm]{./img_chap6/img630b.png}
      \subcaption{Control block diagram of each platform stage. Left figure is that of the IX stage, right figure is that of the EX.}\label{img:img630b}
    \end{center}
  \end{minipage}
  \caption{The baseline compensation control of each platform stage for demonstration.}
\end{figure}

To demonstrate the baseline compensation system using GIF, we design a simple control configuration. Although the simplest configuration is the feedforward using the GIF, the feedforward control cannot suppress the disturbances other than the horizontal seismic noise such as the tilt ground motion or the temperature fluctuation \cite{sekiguchi2016astudy}. Because these disturbances could move the platform stage in a horizontal direction, we need a feedback control using the position sensor to suppress these disturbances. Therefore we use the sensor correction control rather than the feedforward control. 

Figure \ref{img:img630a} shows the schematic control of the platform stage for the input x-arm test mass (IX) and end x-arm test mass (EX). While the IX stage is fed back the relative position sensor signal to the actuator on the stage, the EX stage is added to the GIF strainmeter signal. In other words, while the IX stage is locked to the local IX ground, the EX stage is also locked to the local EX ground, but this feedback signal is corrected by using the GIF strainmeter. The GIF measure the baseline length changes, which means the differential motion of the IX and EX ground. Therefore, the feedback signal corrected by using GIF is the same as the feedback signal of the IX stage. Thus, the EX stage can follow the IX stage by using the corrected feedback signal.

Figure \ref{img:img630b} shows the control diagram of each stage. In both stages, the displacement of the IX platform stage $X_{\mathrm{STG}}$ is disturbed by the local seismic motion $X_{\mathrm{GND}}$ though the mechanical response of the inverted pendulum (IP) $H_{\mathrm{s}}$. Moreover, the displacement of the IX test mass is also disturbed by this seismic noise through the mechanical response of the pendulum $H_{\mathrm{TM}}$. In order to reduce the test mass motion in the low-frequency region, below 1 Hz, the platform stage is controlled by the feedback control using the relative position sensor. $S_{\mathrm{L}},\, N_{\mathrm{fb}}$ and $B_{\mathrm{L}}$ are the displacement response and the noise of the relative position sensor and the low-pass filter not to inject the sensor noise to the feedback signal. The feedback signal is sent to the actuator, whose transfer function from the actuator force to the platform stage is given by $P_{\mathrm{a}}$, through the control filter $C_{\mathrm{fb}}$. On the other hand, the feedback signal of the EX stage is corrected by the GIF signal.

In this situation, each displacement of the stage are given by 
\begin{eqnarray}
  X_{\mathrm{STG(IX)}} &=& \displaystyle\frac{G}{1+G} X_{\mathrm{GND(IX)}} + \frac{G}{1+G} N_{L} + \frac{1}{1+G} H_{\mathrm{s}} X_{\mathrm{GND(IX)}} \\ \nonumber,
  X_{\mathrm{STG(EX)}} &=& \displaystyle\frac{G}{1+G} \left(1- \frac{C_{\mathrm{sc}}S_{\mathrm{wit}}}{B_{\mathrm{L}}S_{\mathrm{L}}}\right) X_{\mathrm{GND(EX)}} + \frac{G}{1+G}N_{L} \\ \nonumber
  &+& \frac{G}{1+G} \frac{C_{\mathrm{sc}}S_{\mathrm{wit}}} {B_{\mathrm{L}}S_{\mathrm{L}}} X_{\mathrm{GND(IX)}}
  + \frac{G}{1+G} \frac{C_{\mathrm{sc}}S_{\mathrm{wit}}} {B_{\mathrm{L}}S_{\mathrm{L}}} N_{\mathrm{wit}}\\ 
  &+& \frac{1}{1+G} H_{\mathrm{s}} X_{\mathrm{GND(EX)}},
\end{eqnarray}
respectively, where $G=C_{\mathrm{fb}}P_{\mathrm{a}}S_{\mathrm{L}}B_{\mathrm{L}}$ is the loop gain. Here, if $G\gg1$ and we design the sensor correction filter $C_{\mathrm{sc}}$ so that
\begin{eqnarray}
  \frac{C_{\mathrm{sc}}S_{\mathrm{wit}}}{B_{\mathrm{L}}S_{\mathrm{L}}} = 1,
\end{eqnarray}
the displacement of each stage are give as 
\begin{eqnarray}
  X_{\mathrm{STG(IX)}} &=& X_{\mathrm{GND(IX)}} + N_{\mathrm{L}},\\
  X_{\mathrm{STG(EX)}} &=& X_{\mathrm{GND(IX)}} + N_{\mathrm{L}} + N_{\mathrm{wit}}.
\end{eqnarray}
Moreover, if the noise of the GIF, which is the wittness sensor is smaller than that of the relative position sensor, both stage motions are the same each other; $X_{\mathrm{STG(EX)}}=X_{\mathrm{STG(IX)}}$. This same motion means the reduction of the differential stage motion. Thus, the cavity length is isolated from the differential ground motion, which is the baseline length fluctuation.




\section{Results and Discussion } \label{sec:sec52}
The performance of the baseline compensation system is evaluated when the system is engaged.

\subsection{Results}
\begin{figure}[h]
  \centering
  \includegraphics[width=12cm]{./img_chap6/img610.png}
  \caption{Length change of both X-arm baseline and X-arm cavity when baseline compensation system is turned on or off. At 12 minutes, the control is on.}\label{img:img610}
\end{figure}

Figure \ref{img:img610} shows the length fluctuation of the arm cavity and of the baseline as a reference. At 12 minutes, the baseline compensation system was turned on. Whereas the X-arm cavity length is drifted during the compensation system was off, the drift is removed during the system was on. This drift is comparable to the earth tide. As a result, this system compensated the deformation of the baseline, and the reduction ratio is almost 1/10.

This result also indicates that the RMS amplitude of the X-arm cavity length is reduced. The amplitude spectrum density of the length when both the compensation system was on and off is shown in Figure \ref{img:img611}. It is clear that the accumulated RMS amplitude is reduced due to the compensation system. In the next, we compare this measured data with the rigid body model.

\begin{figure}[h]
  \centering
  \includegraphics[width=10cm]{./img_chap6/img611.png}
  \caption{ASDs of X-arm caivty length when baseline compensation system is turned on and off. }\label{img:img611}
\end{figure}

\subsubsection{Comparison with the model}
Compare with the measured data and rigid body model of the KAGRA suspensions \cite{sekiguchi2016astudy}. Because this model outputs the state-space model, we can calculate the transfer function. For example, the transfer function from the ground motion to each stage; the platform stage, test mass, and so on.

To simplify the discussion, suppose the CMRR is large enough to ignore the coupling from the common motion to the differential motion, as described in \cref{sec532}. It is a valid assumption below the eigenfrequency of the suspensions. According to Eq.(\ref{eq:eq518}), the transfer function from the differential input to differential output is given by a single transfer function. Therefore, the differential transfer functions from the differential ground motion to the differential output of the stage and the test mass motion are given by the single transfer function of that, respectively.

\begin{figure}[p]
  \begin{minipage}{14cm}
    \centering
    \includegraphics[width=9cm]{./img_chap6/img612.png}
    \subcaption{Noise budget when the compensation system is OFF. Measurement is same as the black line in Fig.\ref{img:img611}. Total is the summation of all the noise contributions.}\label{img:img612} %\hfill\vspace{5pt}
  \end{minipage} 
  \begin{minipage}{14cm}
    \centering
    \includegraphics[width=9cm]{./img_chap6/img613.png}
    \subcaption{Noise budget when the compensation system is ON. Measurement is same as the red line in Fig.\ref{img:img611}. Total is the summation of all noise contributions assuming the reduction factor of sensor correction of 1/20.}\label{img:img613}
  \end{minipage}
  \caption{Comparison between the measurement of X-arm and expected value of that. The expected total value is the summation of some noise contribution, which is named noise budget. }
\end{figure}

Figure \ref{img:img612} shows the amplitude spectrum densities (ASDs) of the X-arm cavity length when the compensation system is OFF. The black line is the ASD calculated by the feedback signal of the X-arm cavity. The red line is the ASD, which is the summation of the noise contributions, the noise of the relative position sensor, named LVDT (blue line), and the noise of the differential baseline length change measured by the GIF strainmeter (orange line). Above 1 Hz, the X-arm cavity length and the seismic noise contribution are not the signals due to the noises of the instruments. Below 1 Hz, the measurement is consistent with the estimation.

Figure \ref{img:img613} shows the ASDs of the X-arm cavity length when the compensation system is ON. The red line, which indicates the summation of the noise contribution estimated by the rigid body model, is calculated assuming the reduction factor of the sensor correction of 1/20, as mentioned in \cref{sec:sec513}. This reduction factor is calculated from the relative calibration error of 5 \% between the LVDT and GIF. Although this reduction rate should be realized, the measurement is not consistent with the estimation assumed the reduction rate. The RMS of the cavity length fluctuation is limited by peaks around 200 mHz.

\subsection{Discussion}
The peaks around 200 mHz, which are the main contributions to the RMS, are correlated with the other degrees of freedoms (DOFs). 

Figure \ref{img:img614} shows the ASDs in the top figure and the coherence in the bottom figure when the compensation system was off. In the top figure, the ASDs of the X-arm cavity length and the baseline length changes are displayed. The baseline length changes are shown by two ASDs; the length change measured by the GIF strainmeter and that given by the differential signal of two seismometers, which is installed near the IX and EX stages. While, above 1 Hz, the baseline length change should be referred by the seismometer differential signal, below 50 mHz, the length change should be referred by the GIF strainmeter signal because of this self-noise. One can find that the X-arm cavity length is enhanced by some mechanical peaks compared with the baseline length change. On the other hand, the bottom figure shows some coherence between the X-arm cavity length and the GIF, and between the cavity length and the other DOFs' signals; the feedback signals of the yaw and transverse directions on the each IX and EX platform stages, which controls are needed to keep the X-arm cavity on resonance. Whereas the cavity length has a coherence with the deformation of the baseline measured by GIF strainmeter (blue) around 0.2 - 0.7 Hz broadly, coherence with the other DOFs does not exist clearly in this frequency region. This coherence implies that the cavity length is mainly disturbed by the deformation of the baseline.

Figure \ref{img:img615} show the ASDs and coherence when the compensation system was on. Around 0.2 Hz, the coherence between the cavity length and the many other DOFs appear, although these coherences did not when no length compensation. These coherences imply the cavity length is disturbed by the internal DOFs coupling.

\begin{figure}[p]
  %\begin{minipage}{14cm}
    \centering
    \includegraphics[width=15cm]{./img_chap6/img614.png}
    \caption{Coherence between the cavity length and GIF strainmeter, other degrees of freedoms on the stage control when the compensation system is on. (Top) ASD of the cavity length and baseline length. (Bottom) The coherence between the cavity's length and some signals.}\label{img:img614}
    %\end{minipage}
\end{figure}
\begin{figure}[p]
  %\begin{minipage}{14cm}
    \centering 
    \includegraphics[width=15cm]{./img_chap6/img615.png}
    \caption{Coherence between the cavity length and GIF strainmeter, other degrees of freedoms on the stage control when the compensation system is off. (Top) ASD of the cavity length and baseline length. (Bottom) The coherence between the cavity's length and some signals.}\label{img:img615}
  %\end{minipage}  
\end{figure}



\section{Summary of the Chapter} \label{sec:sec53}
In this chapter, the following items are described:
\begin{itemize}
\item Experimental arrangement for evaluation of the X-cavity length fluctuation was described.
\item As a result, above 1 minutes period, the fluctuation is reduced by 20 dB, while below this period, the fluctuation is reduced by 6 dB.
\item According to the coherence measurement, the internal coupling to the cavity length would limit the performance in the short period region.
\end{itemize}


 % Demonstration of Arm Length Compensation
\chapter{Conculusion and Future Prospects} \label{chap6}
In this chapter, we conclude the thesis and describe future prospects.

\section{Conclusion}
In this study, we developed the baseline compensation system for reducing the influence of the low-frequency seismic noise. The baseline compensation system is the active baseline seismic isolation system for the optical arm cavities. This system only uses the GIF which can measure the baseline length changes directly. The conventional system uses a seismometer to measure the length change, which means that the insufficient isolation performance below 0.1 Hz due to the sensitivity of the seismometer.

We evaluated the performance of the new system by using the X-arm cavity. As a result, we have shown the reduction of the length at least by -6 dB below 1 Hz. The reduction was also in the earth tides band at least by -40 dB. This reduction is the first result of the kilo-meter scale active baseline seismic isolation in the world.

\section{Future Prospects}
In order to obtain higher seismic isolation performance, a seismic isolation system that combines the conventional seismic isolation system and the baseline length compensation system developed in this study is useful.

In this section, we describe the improvement of the baseline compensation system in the future and the prospection of improvement of the duty cycle.

\subsection{Control Design} \label{sec:444}
The demonstration in this study was for 30 minutes, during which the microseismic noise was quiet. In order to improve the duty cycle, it is necessary to isolate the seismic noises even in bad weather. Although we did not implement in this study, it is useful to incorporate a conventional active inertial seismic isolation system so that seismic isolation can isolate the microseismic. In other words, passive seismic isolation using a pendulum is used for frequencies above approximately 1 Hz, a baseline length compensation system is used for frequencies below 1 Hz, and an active inertial vibration isolation system is used for frequencies between 0.1 Hz and 10 Hz. Inertial sensors have higher noise levels than GIFs below 0.1 Hz, but lower levels above that. Therefore, in the microseismic band (0.1 Hz to 1 Hz), the vibration isolation performance of the conventional active inertial vibration isolation system is higher. This conventional system has a narrow frequency band for vibration isolation, but unlike the baseline length compensation system, the common component of the mirrors of the arm cavity can also be isolated. Moreover,  in the microseismic band, the resonance frequency of inverted pendulum exists in the case of KAGRA. Thus the active inertial vibration isolation system using the feedback control is also useful in the point of appropriate damping control of these.

In order to improve the performance of the baseline compensation system, especially the microseismic noise band, we should use the active inertial seismic isolation system to enhance the performance in this band. 

%% \begin{table}[h] 
%%   \begin{center}
%%     \caption{hoge}\label{tb:71}
%%     \begin{tabular}{llllll} 
%%       \hline      
%%       Band [Hz]& $<0.01$ & 0.01 to 0.1 & 0.1 to 1& 1 to 10 & $>10$ \\ \hline \hline
%%       Passive pendulum                         & a & b & c & d & e \\
%%       Active Inertial Seismic Isolation System & a & b & c & d & e \\
%%       Baseline Compensation System             & a & b & c & d & e 
%%     \end{tabular}
%%   \end{center}
%% \end{table}


\begin{figure}[h]
  \begin{center}   
    \includegraphics[width=10cm]{./img_chap5/img511.png}
    \caption{The block diagram of the future baseline compensation system.} \label{img:img511}
  \end{center}
\end{figure}

To simplify the discussion, we supposed the CMRR is large enough to ignore the common motion coupling. Thus, we can just consider the only differential component of the motion in this system.

The control diagram of the future baseline compensation system can be represented as shown in Figure \ref{img:img511}. Essentially, all the terms in this figure are the same as the active inertial isolation system shown in Figure \ref{img:img503} except the input and output signals. These signals are replaced as $X_{\mathrm{d}}$ and $Y_{\mathrm{d}}$, which are the differential displacement of the ground and platform stage motions, respectively. In this figure,  $S_{\mathrm{wit}}$ and $N_{\mathrm{ff}}$ are the frequency response and the self-noise of the GIF, respectively. Furthermore, the noises $N_{\mathrm{H}}$ and $N_{\mathrm{L}}$ are multiplied by $\sqrt{2}$ in the case of the amplitude unit.

The displacement of the differential motion of the stages is given by
\begin{eqnarray}\nonumber
  Y_{\mathrm{d}} &=&\frac{G}{1+G}L\Delta_{\mathrm{sc}} X_{\mathrm{d}} + \frac{1}{1+G} \Delta_{\mathrm{ff}} X_{\mathrm{d}}\\ \nonumber
  &+& \frac{G}{1+G}\left(HN_{H}+LN_{L}\right) + \frac{G}{1+G}C_{\mathrm{sc}}S_{\mathrm{wit}}N_{\mathrm{ff}} \\ 
  &+& \frac{1}{1+G}P_{\mathrm{a}} C_{\mathrm{ff}}S_{\mathrm{wit}}N_{\mathrm{ff}} \label{eq:eq520}.
\end{eqnarray}
According to the first and second terms in the equation, the contribution of the ground motion $X_{\mathrm{d}}$ can be independently isolated by three factors: $\Delta_{\mathrm{sc}}$, $\Delta_{\mathrm{ff}}$, and $L$. While the current design needed high loop gain $G$ to reduce the second term, this design does not need due to the $\Delta_{\mathrm{ff}}$. Thus, the feedforward path can make flexible design. Furthermore, the low-pass filter $L$ can isolate the lower frequency motions if the inertial sensor on the stage has enough sensitivity to measure the microseismic noise band.

\subsection{Improvement of Duty Cycle}
If the baseline length compensation system developed in this study is applied to the currently operating gravitational wave detector, it will be possible to improve the lock loss caused by low-frequency ground vibration.

As shown in Figure \ref{img:img190} in chapter \cref{sec:duty}, the low-frequency ground vibration caused by environmental changes causes an average of 10\% lock loss. This method can improve this. Furthermore, because we also reduce the amount of lock loss, we would reduce the number of lock acquisitions. Thus,  locking time, which takes up 20\% of the unobserved state, can also be reduced. Therefore, our compensation system will improve the lock loss except the maintenance or commissioning. The gravitational-wave detectors whose duty cycle is improved will enhance the gravitational-wave astronomy.
 % Conculusion and Future Directions
\appendix
%% CLIOでの弾性定数測定によると、P波(Longitudinal Wave)とS波(Transverse Wave)それぞれの速度は、$v_{L}=5.54\pm{0.05} \mathrm{km/sec},v_{T}=3.05\pm{0.05} \mathrm{km/sec}$であった。さらに鉱山会社から報告されている岩石の密度$\rho=2.7 \mathrm{g/cm^3}$に基づくと、表\ref{table:table_1}のように弾性定数が得られる。%\cite{竹本2003}
%% \begin{table}[h]
%%   \caption{CLIOサイトの弾性定数}
%%   \label{table:table_1}
%%   \centering
%%   \begin{tabular}{cll}
%%     \hline    
%%     弾性定数 & 計算値 & 単位  \\
%%     \hline
%%     ラメ定数 $\lambda$ & $3.27\times10^11$ & $\mathrm{dyn/cm^2}$  \\
%%     剛性率 $\mu$ & $2.51\times10^11$ & $\mathrm{dyn/cm^2}$  \\
%%     ポアソン比 $\sigma$ & $0.283$ &   \\
%%     \hline
%%   \end{tabular}
%% \end{table}

 % Theory
\bibliography{reference}
\bibliographystyle{unsrt}
\end{document}
